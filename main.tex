%\documentclass{jjsce}
\documentclass[jscefinal]{jjsce}% 登載可決定後の最終原稿を提出するときはこちら → ページ番号なしになる

\usepackage{amsmath}
\usepackage{amsthm}
\usepackage[defaultsups]{newtxtext}
\usepackage[varg]{newtxmath}
\usepackage{bm} % 数式でボールドイタリックを使いたいとき
%% graphicx 利用の場合は以下のどちらかを選ぶ
\usepackage[dvipdfmx]{graphicx}% (u)pLaTeX + dvipdfmx の場合
%\usepackage{graphicx}% LuaLaTeXの場合
%\usepackage[dvips]{graphicx}
%\usepackage{xcolor}
\usepackage[superscript]{cite}
\usepackage{url}
\usepackage{endnotes}
\usepackage[savepos]{zref}
%% hyperref 利用の場合は以下のどちらかを選ぶ
\usepackage[dvipdfmx]{hyperref}% (u)pLaTeX + dvipdfmx の場合
\usepackage{pxjahyper}% (u)pLaTeX + dvipdfmx の場合はこれを併用する
%\usepackage[luatex,pdfencoding=auto]{hyperref}% LuaLaTeXの場合
\aboveEtitlesep20mm % 最終ページの英文タイトル部:本文末の並行止めした後のスペース調整.詳細はreadme.pdf参照.

\usepackage{jjsce-macros}

\newcommand{\fat}[1]{\boldsymbol{#1}}
\begin{document}
\jtitle{
	き裂による散乱波の反響環境下での\\時間反転集束}
%\jsubtitle{}
\etitle{
	TIME-REVERSAL FOCUSING OF ELASTODYNAMIC SCATTERED WAVES 
	FROM A CRACK IN A REVERBERATING ENVIRONMENT}
%\esubtitle{}
\authorlist{%
 \authorentry{木本 和志}{Kazushi KIMOTO}{OU}
 \authorentry{斎藤 隆泰}{Takahiro SAITOH}{GU}
}
\Caffiliate[OU]{正会員 岡山大学大学院 学術研究院環境生命科学学域
(\jipcode{700--8530}岡山市北区津島中3-1-1)}{kimoto@okayama-u.ac.jp}
\affiliate[GU]{正会員 群馬大学大学院 理工学府環境創生部門
(\jipcode{376--8515}群馬県桐生市天神町1-5-1)}{t-saitoh@gunma-u.ac.jp}

%\Jbreakauthorline{4}
%\breakauthorline{4}
%\received{2022}{6}{24} Received Date を記入
%\accepted{2022}{4}{28} Accepted Date を記入
\begin{abstract}
本研究は,時間反転集束の原理に基づくイメージングが,超音波が反響
する状況での探傷にも有効であることを,実験と数値シミュレーション
で示したものである.実験ではT継手形状を模擬した試験体を用い,
人工き裂からの散乱エコーを計測した.数値シミュレーションでは,
時間反転した計測波形を差分法解析の境界値として与え,時間反転場の
集束挙動を調べた.その結果,計測波形上では識別困難なき裂エコーが,
時間反転場において散乱源に集束する様子が明瞭に追跡できることを示した.
さらに,計算で得た時間反転場を利用した超音波イメージングを行い,
散乱源位置を正確に示す画像が得られることを示した.以上により,
時間反転集束法が,継手のような多重反射や回折波が混在する状況での
超音波探傷にも有効であることを明らかにした.
\end{abstract}
\begin{Eabstract}
In this study, ultrasonic measurements and simulations are performed to show 
that the time-reversal imaging works well 
%in the context of ultrasonic testing involving 
with the reverberating wave fields and hence useful for ultrasonic testings. 
In the experiment, ultrasonic echoes from 
an artificial crack in a T-joint specimen are measured. The measured waveforms 
are then time-reversed and re-emitted into a finite difference numerical 
model for the investigation of focusing in a reverberating environment.
It is found as a result that the path of the scattered waves can be traced 
in the time-reversed field back to its origin though their presence is hardly 
recognizable in the original signals. Moreover, the scattering source 
can be located accurately in the reverse-time migration images, 
thus highly robust nature of the time-reversal focusing concept has 
been demonstrated in a ultrasonic testing context.
\end{Eabstract}
\begin{keyword}
time-reversal, ultrasonic nondestructive testing, crack, scattering, imaging
\end{keyword}
\maketitle
\section{はじめに}
	花崗岩はき裂性の結晶質岩で,新鮮な状態でも造岩鉱物粒程度のスケールでは多数のマイクロクラックを有することが知られている\cite{Kudo1}$^-$\cite{Takagi}.
マイクロクラックは花崗岩の透水性や物質輸送特性,剛性,強度,物理・化学的な風化の進行に影響を与える.
そのため,マイクロクラックの大きさや量を把握することは,例えば,高レベル放射性廃棄物の地層処分のように,
長期に渡って安全性を担保する必要のある処分場を花崗岩体中に建設する際に重要となる.
%
岩石中のき裂の大きさや量,分布状況は偏光顕微鏡で岩石薄片を観察することで調べられる.
しかしながら,岩石薄片の作成には手間がかかる上,岩石内部の状態やき裂の経時的な変化を観察することはできない.
一方,岩石の弾性波試験は結果の定量的解釈が難しいことが多い点に難があるものの,試料内部の状態を簡単な検査で調べることができ,
経時的なモニタリングにも適用できるという面で利点がある.
実際,弾性波の伝播はき裂の影響を強く受けるため,弾性波計測結果からき裂の量や進展に関する情報を得る
試みが行われている\cite{Griffiths}.
ただし,花崗岩を始めとする岩石中の弾性波伝播は鉱物粒やき裂による多重散乱のために非常に複雑で\cite{Sato},
弾性波計測データから鉱物粒径やき裂を定量的に評価する方法は確立されていない.
%
花崗岩中の弾性波伝播特性を理解するためには,多数の鉱物粒から成る多結晶質体モデルを用いて
弾性波の多重散乱解析を行うことが有効となる\cite{FEM}.
そのような多重散乱解析の実施には造岩鉱物の密度と弾性係数値を与える必要がある.
これらの物性値のうち密度は,物質の結晶構造と構成原子の質量から鉱物種ごとに
定めることができ,鉱物種間での差も小さい.
各種造岩鉱物の弾性係数も鉱物学分野で古くから調べられており,文献値\cite{AGU}や
分子シミュレーション\cite{Kawamura}の結果から知ることができる.ただし,それらはき裂の無い
単結晶鉱物に対するものであるため,岩石を構成する鉱物粒の弾性係数としてそのまま用いることはできない.
また,弾性係数は結晶軸方向やき裂の有無による不確実性があるため,対象とする岩石の状態を反映して
確率的に与える必要もある.

花崗岩の弾性波速度や弾性係数に関する研究は,これまでに多数行われている.
例えば,工藤ら\cite{Kudo2},佐野ら\cite{Sano1}は花崗岩コア試料の超音波透過試験を行い,
花崗岩が直交異方性体として振る舞うことを示し,マイクロクラックとの関係を検討している.
ただし,それらの研究では岩石コアスケールでのマクロな異方性や弾性波速度が論じられており,
鉱物粒スケールでの弾性係数や弾性波速度を調べたものではない.
これに対してSivaji\cite{Sivaji}およびFukushima\cite{Fukushima}らは,
レーザードップラー振動計を使って花崗岩試料表面に励起された
超音波振動を可視化し,鉱物粒径と弾性波到達時間のゆらぎの関係を調べている.
岩石のランダムな不均質性に起因した弾性波到達時間のゆらぎについてはRythov近似に基づく方法で
理論的な検討が加えられ,到達時間のゆらぎと不均質性の空間スケールの関係が得られることが示されている
\cite{Muller}$^-$\cite{Baig}.
ただし,これらの研究において、弾性波速度の鉱物粒スケールでの変動は,ガウス分布や指数分布
等、取扱が容易なよく知られた確率分布に従うことが仮定されており,造岩鉱物粒の弾性波速度が従う
実際の確率分布はこれまで調べられていない.

そこで本研究では,超音波計測によって局所的な縦波(P波)の位相速度を統計的に求めることを試みる.
これを多数の鉱物粒について鉱物種毎に行うことで,P波位相速度が従う確率分布を評価する.
以下では,この目的のために行った超音波計測方法について述べた後,実験で得られたP波位相速度分布
の結果を報告する.具体的には,1$\sim$2MHzの周波数帯域におけるP波位相速度の頻度分布を示し,
結晶粒のスケールでは鉱物種に依らず速度分散はほとどんど無いこと,
長石と石英ではP波位相速度の分布が有意に異なることを示す.
また,これら実験で得られたP波位相速度を用いた数値波動解析を行い,縦波平面波の散乱減衰挙動
について検討を行う.その結果,P波位相速度の頻度分布を単一の正規分布と仮定した場合と,
計測から得た頻度分布を用いた場合で散乱減衰の挙動が異なることを示す.
以上の結果から,弾性波媒体として花崗岩をモデル化する上で,実際に計測を行って評価した
位相速度の分布を反映したモデルを用いることが必要であることを述べる.
なお,以下ではP波位相速度を単に位相速度と呼ぶ.


\section{超音波エコー計測}
	Fig.1に超音波エコー計測の概要を示す.計測にはT溶接継手の形状を模擬した試験体を用いた.Fig.1はその断面図を示したもので紙面奥行き方向には一様な形状をもつ.試験体を構成する水平材(フランジ)の厚さは12mm,鉛直材(ウェブ)の厚さは7.8mmで,継手部分には隅肉溶接での接合を想定し,余盛りとルートギャップを表現した箇所を設けてある.なお,ここでは,継手の形状だけを模擬することを意図し,試験体はアルミニウムブロックを切削加工して作成している.Fig.2は継手近傍の詳細を示したもので,余盛り表面は半径20mmの円弧とし,脚長7mmとなっている.き裂はルートギャップから発生し,フランジ側に進展するケースを考え,長さ4mm,幅0.2mmのスリットを,鉛直方向から傾き15度の方向に放電加工で作成した.なお,き裂の起点はルートギャップを模擬した1mm角のスリット角部とした. 超音波の送受信はウェブ右側,フランジの上面からのみ可能と仮定し,Fig.1のように送受信センサーを配置した.これは,TOFD法を適用することのできない,制約の厳しい探傷条件を想定したものである.ここで,送信には屈折角70度,公称中心周波数5MHzの圧電探触子を,受信にはレーザードップラー振動計(LDV)を用いた.送信探触子は,溶接始端部から40mmの位置に,探触子前縁部が来るように設置した.一方,LDVによる受信は,止端部から0.2mmの間隔で201点,40mmの範囲で行った.その際,サンプリング周波数は80MHz,平均化回数は4,096回とし,送信探触子は振幅300V,幅0.1sの矩形パルスで駆動した.

Fig.3は,このようにして計測した201点での超音波波形を示した走時プロットである.横軸は経過時間(s)を,縦軸は位置 (mm)を表し,対応する時空間点でのエコー強度(mV)をカラーマップで示している.Fig.3において右下がりの軌跡は0方向への,右上がりのものは0方向への進行波を表す.き裂からの後方散乱波は右方向への進行波として観測される.しかしながら,右方向への進行波にはルートギャップや溶接止端部等からの形状エコーが含まれるため,進行方向や到達時間だけから,き裂エコーを識別することは困難である. 
\begin{figure}[htb]
\centering
%\rule{20mm}{20mm}
	\includegraphics[clip,scale=0.5]{Figs/samples.eps}
%\includegraphics[clip,scale=0.5]{Figs/.eps}
\caption{図のキャプションは図の下に置く}
\label{fig:sample}
\end{figure}


\section{時間反転場の集束挙動}
	時間反転集束法では,観測波形を時間反転して媒体へ入力し,時間を遡る方向へ散乱波の伝播を調べる.
これを実験や数値シミュレーションで行う場合,時間反転波形をどのような物理量として入力するか
考える必要がある.ここでは,前節の観測波形を時間反転し,表面力として数値解析モデルに与える.
これは,フルウェーブインバージョン(full waveform inversion: FWI)における随伴場を
計算することに相当する.そこで本節では,動弾性問題の基礎方程式とFWIにおける随伴問題の
定式化をはじめに示す.次に,観測波形を境界値に用いた随伴問題の解析結果を示し,
散乱源への波動場の集束状況を調べる.数値波動解析は全て二次元(面内波)問題として行い,
ベクトルやテンソル成分は総和規約を適用した表現を用いる.以下では,随伴問題の解である
随伴場を,より物理的意味を想起しやすいという面を重視し,時間反転場と呼ぶことにする.
\begin{figure}[tbh]
\centering
	\includegraphics[clip,scale=0.55]{Figs/sim_model.eps}
	\caption{解析対象領域.各部の寸法はスリットが無いことを除き
	{\bf 図}-\ref{fig:model}の試験体断面と同じ.}
	\label{fig:fd_model}
\end{figure}
\subsection{動弾性問題の基礎式}
{\bf 図}-\ref{fig:fd_model}に示すような,試験体形状に合わせた領域$G$,
時間範囲$T=(0,\, t_f)$における超音波の伝播・散乱問題を,均質な線形弾性体における
動弾性問題として考える.ここで,位置ベクトルを$\boldsymbol{x}=(x_1,x_2)$,時間変数を$t$とし,
物体力は働かず材料減衰は無視できるとする.
%なお,座標系は原点位置を含め実験と一致させ$(x,y)=(x_1,x_2)$とし,
%指標を使った成分表記が必要である場合を除き$(x,y)$で書く.
いま,応力テンソルを$\sigma_{ij}$, 速度ベクトルを$v_i$,
質量密度を$\rho$とすれば,運動方程式は
\begin{equation}
	\rho \dot{v}_i=\sigma_{ji,j}, \ \ (\boldsymbol{x},t)\in G\times T
	\label{eqn:}
\end{equation}
と表される.ただし$(,)$は空間の,$\dot{()}$は時間の微分
を表す.速度$v_i$は変位ベクトル$u_i$の時間微分で与えられ,
ひずみテンソル$\varepsilon_{ij}$は,変位を使って
\begin{equation}
	\varepsilon_{ij}=\frac{1}{2}(u_{i,j}+u_{j,i})
	\label{eqn:FWD}
\end{equation}
と表される.ここで,弾性係数テンソルを$C_{ijkl}$とすれば,フック則は
\begin{equation}
	\sigma_{ij}=C_{ijkl}\varepsilon_{kl}
	\label{eqn:}
\end{equation}
と表され,等方性体では,ラメ定数$\lambda$と$\mu$を用い
\begin{equation}
	C_{ijkl}=\lambda \delta_{ij}\delta_{kl} +
	\mu (
	\delta_{ik}\delta_{jl}
	+
	\delta_{il}\delta_{jk} 
	)
	\label{eqn:Cijkl_iso}
\end{equation}
とすることができる.このとき,縦波および横波の位相速度は,それぞれ,
\begin{equation}
	c_{L}=\sqrt{\frac{\lambda + 2\mu}{\rho}}
	, \ \ 
	c_{T}=\sqrt{\frac{\mu}{\rho}}
	\label{eqn:}
\end{equation}
となる.$t=0$を超音波の送信時刻とすれば,初期条件は
\begin{equation}
	u_i(\boldsymbol{x},0)=0, 
\ \ v_i(\boldsymbol{x},0)=0, 
\ \ \boldsymbol{x} \in G
	\label{eqn:IC}
\end{equation}
としてよい.超音波の励起は領域境界$\partial G$の一部である$S$に加えた
表面力$\bar{t}_i^{(n)}$で行われ,それ以外の箇所$\partial G\setminus S$で
表面力はゼロとする.このとき,境界条件は次のように表される.
\begin{equation}
	t_i^{(n)} = \sigma_{ji}n_j=
	\left\{
		\begin{array}{ll}	
			\bar{t}^{(n)}_i, & (\fat{x},t)\in S\times T \\
			0,  & (\boldsymbol{x},t)\in \left( (\partial G \setminus S)\times T\right) 
		\end{array}
	\right.
	\label{eqn:BC}
\end{equation}
なお,$n_i$は外向きの単位法線ベクトルを表す.
\\
\hspace{\parindent}
以上の定式化に関する留意事項を述べる.
第一に,解析領域$G$は継手周辺部分の試験体断面を型どったもので,
試験体全体をモデル化したものではない.
第二に,数値解析モデルにはスリットは含まれない.
これは,スリットの無いモデルでの計算結果からスリットの検出
や画像化を行うことがここでの目的となるためである.
最後に,領域$S$に加えられる表面力$\bar{t}_i^{(n)}$は,実際には超音波探触子と試験体の
接触力によって生ずるもので,直接測定できない.
そこで,超音波探触子から入射される波動場を再現する際には,$\bar{t}^{(n)}_i$を
%
境界条件として
%
与えた計算を行うのではなく,プレート裏面側で計測した速度波形を使った計算を行う.
%この点については,第4節であらためて説明する.
その具体的な方法については第4節で説明する.
%%%%%%%%%%%%%%%%%%%%%%%%%%%%%%%%%%%%%%%%%%%%
\subsection{随伴問題(時間反転場の支配方程式)}
超音波探傷で検出すべき欠陥を弾性係数や密度が周囲と異なる領域と
みなせば,$C_{ijkl}$や$\rho$の分布を推定することで欠陥が検出できる.
FWIはこれらのモデルパラメータ:
\begin{equation}
	\fat{m}=\left( \rho, \, C_{ijkl}\right)
	\label{eqn:mprms}
\end{equation}
を推定するために利用することのできる方法の一つである.
FWIでは,観測波形$u_n^{meas}$と,仮定したモデルパラメータ$\fat{m}$
に対して得られるシミュレーション波形$u_n$の差を
% 目的関数$\chi(\fat{m})$と
最小化するよう$\fat{m}$を繰り返し修正して推定する.
%
モデルパラメータの修正には,目的関数$\chi(\fat{m})$を定義し,その勾配
$\nabla \chi(\fat{m})$を用いる.以下に述べるように,$\nabla \chi(\fat{m})$は
時間反転場を用いて表されるFr$\acute{\rm e}$chet Kernelから計算できる.
本研究のRTMイメージングでは,Fr$\acute{\rm e}$chet Kernelを画像化関数として用いる.
%
いま,未知欠陥が領域$G$に含まれるとし,
境界$\partial G$の一部$R\in\partial G$で観測を行い,次のデータを得たとする.
\begin{equation}
	\left\{ 
	\left. 
	u_n^{meas}(\boldsymbol{x},t)\right|  (\boldsymbol{x},t)\in R\times T
	\right\}
	\label{eqn:data}
\end{equation}
ただし,添字$n$は$u_n=u_in_i$,すなわち変位の法線方向成分を表す.
ここで,動弾性問題の解として得られた$R$上でのシミュレーション波形を
$u_n(\boldsymbol{x},t)$とし,FWIにおける目的関数を
\begin{equation}
	\chi(\fat{m}):= \frac{1}{2} \int _T\int _R \left|u_n-u_n^{meas}\right|^2 dtdx_1
	\label{eqn:cost}
\end{equation}
としてみる.
このとき,$\chi(\fat{m})$の$\delta \fat{m}=(\delta \rho,\, \delta C_{ijkl})$
方向への次の意味での微分
\begin{equation}
	\nabla \chi (\fat{m})\delta \fat{m} = \lim_{\varepsilon \rightarrow 0}
	\frac{1}{\varepsilon}
	\left\{
		\chi(\fat{m}+\varepsilon \delta \fat{m})
		-
		\chi(\fat{m})
	\right\}, 
	\label{eqn:del_chi}
\end{equation}
は,
\begin{equation}
	\nabla \chi(\fat{m}) = 
	-\int_G K_{\rho}(\fat{x})\delta \rho dv
	- 
	\int_G K_{ijkl}(\fat{x}) \delta C_{ijkl} dv
	\label{eqn:}
\end{equation}
の形で表される.
%
なお,$(\rho,C_{ijkl})$はモデルパラメータの現在の推定値を,
$(\delta \rho, \delta C_{ijkl})$は修正量であることを意味する.
%
$K_\rho(\fat{x})$と$K_{ijkl}(\fat{x})$はFr$\acute{\rm e}$chet 
kernelと呼ばれ,随伴問題の解$u^{\dagger}(\fat{x},t)$を用いて次のように与えられる\cite{Fichtner}.
\begin{eqnarray}
	K_\rho(\fat{x}) &= & \int _T \dot{u}_i \dot{u}^{\dagger}_i dt 
	\label{eqn:K_rho}
	\\
	K_{ijkl}(\fat{x}) &= & \int _T u_{i,j} u^{\dagger}_{k,l} dt 
	\label{eqn:K_ijkl}
\end{eqnarray}
ここで,$u_i^{\dagger}$を解にもつ随伴問題は,次の終端値-境界値問題として
定式化される.
\begin{itemize}
\item
支配方程式:
\begin{equation}
	%\rho \ddot{u}_i^{\dagger} =
	\rho \ddot{u}_i^{\dagger}(\fat{x},t) =
	%\sigma^{\dagger}_{ji,j}, \ \ 
	\sigma^{\dagger}_{ji,j}(\fat{x},t), \ \ 
	(\fat{x},t) \in G\times T
	\label{eqn:wveq_adj}
\end{equation}
ただし,
\begin{equation}
	%\sigma_{ij}^{\dagger}= C_{ijkl}u_{k,l}^{\dagger}.
	\sigma_{ij}^{\dagger}(\fat{x},t)= C_{ijkl}u_{k,l}^{\dagger}(\fat{x},t).
	\label{eqn:sigma_dgg}
\end{equation}
\item 
終端条件:
\begin{equation}
	u_i^{\dagger}(\fat{x},t_f) =0,  \ \
	\dot{u}_i^{\dagger}(\fat{x},t_f) =0,  \ \, \fat{x}\in G
	\label{eqn:IC_adj}
\end{equation}
\item
境界条件:
\begin{equation}
	t_i^{(n)\dagger}
	=
%	\sigma_{ji}^{\dagger}(\fat{x},t)n_j(\fat{x}) =
	\left\{
		\begin{array}{ll}
			(
			u_n^{meas}(\fat{x},t)
			-
			u_n(\fat{x},t)
			)n_i(\fat{x}), & \fat{x}\in R \\
			0, & \hspace{-3mm} \fat{x} \in \partial G\setminus R
		\end{array}
	\right.
	\label{eqn:BC_adj}
\end{equation}
ただし,$t_i^{(n)\dagger}=t_i^{(n)\dagger}(\fat{x},t)=\sigma_{ji}^{\dagger}(\fat{x},t)n_j(\fat{x})$.
\end{itemize}
式(\ref{eqn:wveq_adj})-(\ref{eqn:BC_adj})で表される随伴問題は,
時間反転:
\begin{equation}
	\tau=t_f-t
	\label{eqn:tau_def}
\end{equation}
により,$\tau$を新しい時間変数とすれば,弾性波の初期値-境界値問題に帰着される.
ただし,物理的な時間$t$に関して時間を遡る方向に問題を解くことに変わりはないため,
随伴場$u_i^{\dagger}(\fat{x},\tau)$は時間反転場とみなすことができる.
\subsection{時間反転場の数値解析法と計算条件}
時間反転場,すなわち,随伴問題(\ref{eqn:wveq_adj})-(\ref{eqn:BC_adj})の数値解析
には2次元FDTD法\cite{Fellinger1995, FDTD_KK}を用いる.
解析領域は図-\ref{fig:fd_model}とし,実験時の波形観測領域$R$に,
時間反転波形を表面力として加える.ただし,解析領域の打ち切り位置で実験では生じない
反射波が発生することを避けるため,プレートとリブの打ち切り位置には,厚さ15mmのPML吸収領域を設ける.
モデルパラメータであるラメ定数$\lambda,\mu$と密度$\rho$は,
実測したアルミニウム試験体の位相速度
\begin{equation}
	c_L=6.35{\rm km/s}, \ \ c_T=3.15{\rm km/s}
	\label{eqn:phase_vels}
\end{equation}
と,密度の文献値$\rho=2.7{\rm g/cm}^3$を与える.
FDTD法における空間と時間の離散化幅$h$と$\Delta t$は,それぞれ,
\begin{equation}
	h=0.05{\rm mm}, \ \ \Delta t=3{\rm ns}
	\label{eqn:}
\end{equation}
とし,解析時間範囲は$T=(0,t_f)=(0,45)\mu$s,空間範囲は媒体$G$を含む,
幅90mm高さ35mmの矩形領域とした.なお,空間の離散化では,媒体が存在しない
領域も含めスタガード格子が配置しているが,計算時には使用されない.
以上の設定では,
時間ステップ数$15,000$,空間格子数は未知量あたり1,800$\times$700となる.
%
%$t_f=45\mu$s以後の時刻にも,右方向への強い進行波があるが,
%それらはスリットの有無に関わらず現れ,強度も左進行波と同程度
%であることからスリットに起因したものである可能性は低い.
%このことから,ここでは,
%
\\
\hspace{\parindent}
随伴問題の境界条件(\ref{eqn:BC_adj})において,$u_n^{meas}$には実験で得られた
超音波波形を用いる.
\begin{figure}[thb]
\centering
	\includegraphics[clip,scale=0.5]{Figs/kwfilted_xt.eps}
	\caption{観測波形から抽出した(a)左進行波と(b)右進行波.(c)は表面波成分除去後の右進行波.}
	\label{fig:kwfilted_xt}
\end{figure}
%順問題の解である$u_n(\fat{x},t)$には,入射波変位$u_n^{in}$を与えることが望ましい.
%それが可能である場合,$u_n^{meas}$は全変位だから,$u_n-u_n^{meas}$が散乱波となり,
%時間反転場には散乱源に戻る波動場が明瞭に現れる.
順問題の解である$u_n(\fat{x},t)$はスリットがない場合の変位場だから,
これを入射場とすれば$u_n^{meas}-u_n$は散乱波を意味し,
時間反転場には散乱源に戻る波動場が現れると期待される.
%
しかしながら,実測波形にはノイズが含まれること,
本来3次元的な問題を計算コストを抑えるために2次元問題として扱っていることから,
実測波形に含まれる入射波成分を
%
シミュレーション波形を使って
%
正確に差し引くことは難しい.
とりわけLDV計測では,試料の表面状態に応じた受光強度の変化があり,その補正も考慮した
入射波成分の特定は困難である.そこで本研究では,スリットから受信点側に伝わる,右($x_1>0$)
方向進行波を散乱波とみなし,時間反転場の計算に用いる.
すなわち,観測波形$u_n^{meas}(x_1,t)$を,
右進行波$u_n^{+}(x_1,t)$と
左進行波$u_n^{-}(x_1,t)$の和として
\begin{equation}
	u_n^{meas}(x_1,t)=u_n^{+}(x_1,t)+u_n^{-}(x_1,t)
	\label{eqn:split}
\end{equation}
で表し,$u_n^+=-(u_n^{-}-u_n^{meas})$を時間反転場解析の
境界値(\ref{eqn:BC_adj})に用いる.
%
なお,観測面$R$上で$x_2$は一定のため,$u_n^{\pm}$等の引数から$x_2$は省略している.
%
式(\ref{eqn:split})の分離は,観測波形を2次元フーリエ変換して
波数-周波数スペクトル:
\begin{equation}
	U(k,\omega)= \frac{1}{(2\pi)^2} \iint u^{meas}_n(x_1,t)e^{-ikx_1+i\omega t}dtdx_1
	\label{eqn:Ukw}
\end{equation}
を計算し,$k$と$\omega$が同符号のスペクトル成分は$u_n^-$に,
異符号の成分は$u_n^+$に寄与すると考え,帯域制限した次のフーリエ逆変換で求める.
\begin{eqnarray}
	u_n^-(x_1,t) &=& \iint H(-k\omega)U(k,\omega)e^{ikx_1-i\omega t}dk d\omega
	\label{eqn:um} \\
	u_n^+(x_1,t) &=& \iint H(k\omega) U(k,\omega)e^{ikx_1-i\omega t}dk d\omega 
	\label{eqn:up}
\end{eqnarray}
ここで,$H(s)$はヘビサイドステップ関数を表す.
この方法では,入射波成分が大部分を占める左進行波を,
観測波形以外の情報を用いることなく時間反転場解析での
入力波形から取り除くことができる.
{\bf 図}-\ref{fig:kwfilted_xt}の(a)と(b)は,{\bf 図}-\ref{fig:bscans}(a)に示した
速度波形データを$\dot u_n^{meas}(x_1,t)$とし,式(\ref{eqn:Ukw})-(\ref{eqn:up})で計算した
左右進行波成分の速度波形$\dot{u}_n^-(x_1,t),\dot{u}_n^+(x_1,t)$である.
なお,これらのプロットは$\dot u_n^{meas}(x_1,t)$の最大値で規格化されている.
{\bf 図-}\ref{fig:kwfilted_xt}(a)の結果では,右上がりの波群はほとんど見られず,
右進行波$\dot u_n^+$が適切に取り除かれている.
一方,{\bf 図}-\ref{fig:kwfilted_xt}(b)では,
主として右進行波$\dot u_n^{+}$から成ることが確認できるが,
振幅の大きな左進行波の一部が除去しきれていない.
これは,LDV観測における受光強度のばらつきにより,速度波形$u_n^{meas}$に
不規則な空間変動があることが原因として考えられる.
%
なお,{\rm 図}-\ref{fig:bscans}の観測波形では,$50\mu$s以後にも右方向への強い
進行波が現れている.ただしそれらはスリットの有無によらず観測され,
強度も左進行波と同程度であることから,スリットに起因したものである
可能性は低い.このことも考慮し,{\rm 図}-\ref{fig:kwfilted_xt}では,$50\mu$s
までの進行方向分離と表面波成分の除去状況を示している.
%
%%%%%%%%%%%%%%%%%%%%%%%%%%%%%%%%%%%%%%%%%%%%
\subsection{時間反転場の計算結果}
以上の方法で計算した時間反転場の様子を{\bf 図}-\ref{fig:snap_crack_rwv}に示す.
この図は,速度場$| \dot{\fat{u}}^{\dagger}(\fat{x},t)|$のスナップショットを6つの
時刻$\tau$について示したものである.なお,速度振幅は適度なコントラストで波動場が可視化されるように
スケーリングした相対値として表示されている.ただし,スケーリングの基準値は(a)-(f)のプロット全てに
共通であるため,速度振幅の大小関係はプロット間で互いに比較できる.
\begin{figure}[bth]
\centering
	\includegraphics[clip,scale=0.4]{Figs/SnapCrackRwv.eps}
	\caption{時間反転場のスナップショット(スリットあり)}
	\label{fig:snap_crack_rwv}
\end{figure}
(a)の結果をみると,この時刻では,大きな振幅を持った箇所が表面近傍に集中している.
これは,表面波が大きな振幅をもって伝播していることを示す.
\begin{figure}[t]
\centering
	\includegraphics[clip,scale=0.45]{Figs/kwfilted_kw.eps}
	\caption{計測した速度波形の波数-周波数スペクトル}
	\label{fig:kwfilted_kw}
\end{figure}
やや時間が経過した(b)では,止端部$x_1=0$に達した表面波が回折波を励起している.
止端で発生した回折波は,(b)の時刻以後,継手内をSV波として進展する.このような
回折で生じたSV波は次第に空間的に広がる.その一部は散乱源に達するもののいずれか
の点に集束することはなく, スリットの検出や位置特定にはあまり有効でない.
%
そこで,強い回折波の原因となる表面波成分を,入力波形$u_n^{-}(x_1,t)$から取り除いて
時間反転場解析を行う.表面波の位相速度$c_R$は,等方弾性体の場合$c_L$と$c_T$
から理論的に求められる\cite{Schmerr1999}.また,計測波形の波数-周波数スペクトル
$U(k,\omega)$において,表面波成分は傾き$c_R^{-1}$の直線に沿って分布する.
{\bf 図}-\ref{fig:kwfilted_kw}(a)はこのことをみるために,$|U(k,\omega)|$を
計算してプロットしたもので,横軸を周波数$f=\frac{\omega}{2\pi}$,
縦軸を波数$\xi=\frac{k}{2\pi}$とし,スペクトル振幅を最大値で無次元化して
示している.この中で,傾き$1/c_R=1/2.86$(km/s)$^{-1}$の,原点を通る直線付近の
スペクトル成分が表面波で,非常に狭い範囲に集中していることが分かる.
そこで,この直線に沿って平均位置を移動させながら,標準偏差0.2MHzのガウス型
窓関数を周波数軸上で作用させ,表面波成分を取り除く.{\bf 図}-\ref{fig:kwfilted_kw}(b)は
その結果を示したもので,これを時空間領域にフーリエ逆変換することで
{\bf 図}-\ref{fig:kwfilted_xt}(c)のような走時波形が得られる.
この走時波形には,同図(b)に見られた右上がりの直線的な軌跡を示す表面波が概ね取り除かれている.
そこで,表面波を除去した(c)の速度波形を用いて時間反転場を求めると,
{\bf 図}-\ref{fig:snap_crack}のような結果が得られる.
\begin{figure}[thb]
\centering
	\includegraphics[clip,scale=0.4]{Figs/SnapCrack.eps}
	\caption{時間反転場のスナップショット(スリットあり,表面波成分を事前に除去)}
	\label{fig:snap_crack}
\end{figure}
この図に示す(a)の時刻では,時間反転波形の入力域$R$から発生した縦波が
白の矢印で示す方向に進展している.この縦波はプレート下面で反射され,
(b)の時刻では進行方向を変えて余盛部分に向かっている.続く(c)の時刻では,
余盛り表面に縦波が到達して反射され,横波へのモード変換が起きる.
このようにして生じた横波は,(d)で余盛表面を離れてスリットに向かい,
(e)の時刻を経て次第に集束しながら,(f)の時刻において試験体では
スリットがある位置に到達している.
%
なお,ここに示された散乱波の振動回数が6$\sim$7回程度多いことは,
スリットと切り欠きからのエコーが混在するためと考えられる.
%
以上の観察から,この散乱波の起点がスリットにあり,どのような経路を辿って観測
領域$R$に至ったかが明確に理解できる.このような経路と伝播モードの散乱波が,
十分検出しうる強度で観測波形に含まれることを事前に知ることは困難で,
表面波の除去と時間反転場の計算によってはじめて識別が可能になる.
%このことは,予め経路特定が必要となるイメージング方法では,
%ここで見出された散乱波を利用することができず,時間反転法をベースとした
%イメージング法の適用が必要かつ有効であることを示唆する.
%\\ %\hspace{\parindent}
%スリットの無い試験体で得られた観測波形から同じ手順で表面波を除去し,
%時間反転場解析を行った結果を{\bf 図}-\ref{fig:snap_none}に示す.
%この場合,入力域$R$では横波が発生し,(a)の時刻では白の矢印の方向へ進行し,
%(b)の時刻では底面での反射が起きている.続く (c),(d),(e)では横波のまま
%継手内部を上昇し,最終的には(f)の時刻で切り欠き角部に到達する.
%このように,スリットの有無で時間反転場の挙動は明らかに異なり,
%元々の観測波形({\bf 図}-\ref{fig:bscans})の比較では判然としない
%媒体内部の状況が,時間反転場解析を通じて極めて明瞭に提示されていると言える.
%\begin{figure}[htb]
%\centering
%	\includegraphics[clip,scale=0.4]{Figs/SnapNone.eps}
%	\caption{時間反転場のスナップショット(スリットなし,表面波成分を事前に除去)}
%	\label{fig:snap_none}
%\end{figure}

\section{RTMによる時間反転イメージング}
	最後に,reverse-time migration(RTM)の考え方に基づき,きず(スリット)の
超音波イメージングを行った結果を示す.
\subsection{イメージング方法}
RTMでは入射場と時間反転場の相関を
取る形式の画像化関数を定義し,弾性係数や密度のコントラストを再構成する.
例えば,イメージングに速度場を用いる場合,最も単純な画像関数$I(\fat{x})$は,
入射変位場を$u_i^{in}(\fat{x},t)$として次式で与えられる.
\begin{equation}
	I(\fat{x})=\int_T \dot u^{in}_i(\fat{x},t) \dot u_i^{\dagger}(\fat{x},t)dt
	\label{eqn:Ix}
\end{equation}
これは,FWIにおけるFr$\acute{\rm e}$chet kernelの一つ,
式(\ref{eqn:K_rho})の$K_\rho(\fat{x})$において,
$u_i=u_i^{in}$としたものに他ならない.
つまり,この場合のRTMは,FWIにおける目的関数の勾配を画像化関数として
用いていることになる.ただし,RTMでは他にも様々な画像化関数が提案されているため\cite{Jones},
RTMが必ずしもFWIの特別なケースということではない.
本節では,$K_\rho(\fat{x})$に加え,FWIのもう一つのカーネル
$K_{ijkl}(\fat{x})$を用いてイメージングを行う.
$K_{ijkl}(\fat{x})$は4階のテンソルだが,
等方性体の場合,式(\ref{eqn:Cijkl_iso})を用いれば,
\begin{equation}
	\int_G K_{ijkl}C_{ijkl}dv
	=
	\int_G \lambda u_{i,i}u^{\dagger}_{j,j} dv
	+
	\int_G \mu u_{i,j}u^{\dagger}_{i,j} dv
	\label{eqn:}
\end{equation}
となる.つまり,$K_{ijkl}$の独立な成分はラメ定数に対応した二つである.
それらをあらためて
\begin{equation}
	K_\lambda(\fat{x}):= \int_T u_{i,i}u^{\dagger}_{j,j} dt
	\label{eqn:K_lmb}
\end{equation}
\begin{equation}
	K_\mu(\fat{x}):= \int_T u_{i,j}u^{\dagger}_{i,j} dt
	\label{eqn:K_mu}
\end{equation}
と定義する.以下に示すRTMイメージングでは,これら
$K_\rho, K_\lambda$および$K_\mu$において,$u_i(\fat{x},t)=u_i^{in}(\fat{x},t)$
としたものも画像化関数として用いる.
\subsection{入射場の計算}
RTMイメージングに用いた入射速度場$\dot{u}_i^{in}(\fat{x},t)$の挙動を
{\bf 図}-\ref{fig:inc_btm}に示す.これは,{\bf 図}-\ref{fig:spec}の
試験体と同じ厚さ(12mm)のアルミニウム平板を用い,別途計測した速度波形を利用して
FDTD計算で求めたものである.
平板を使った実験は,超音波探触子からの入射場を把握することを目的としたもので,
{\bf 図}-\ref{fig:experiment2}のように,探触子設置面と反対の側からLDV
で表面振動を計測した.
%
探触子からは約70度の角度でSV波が送信されるため,鉛直方向にも十分な強度の
振動成分をもち,LDVでの計測も比較的容易である.
%
LDV計測は{\bf 図}-\ref{fig:experiment2}に示す
幅65mmの範囲$S'$において0.2mmピッチで行い,得られた速度波形を使い
FDTD計算で入射場,すなわちスリットが存在しない場合の波動場を再構成する.{\bf 図}-\ref{fig:fd_model}の$S'$は,
その際,速度波形を与えて加振したFDTDモデル上の位置を表す.
つまり,ここでは,実際の送信源である超音波探触子の
設置面$S$でなく,$S'$に相当する位置で実測した波形を擬似的な
振源項として与え,FDTD法で試験体内部の入射場を求めている.
そのため,探触子から入射された直後に左下方向へ進む波動場はFDTD計算上
再現されず,{\bf 図}-\ref{fig:inc_btm}では,底面で一回反射した
後の入射場の様子が現れている.
\begin{figure}[thb]
\centering
	\includegraphics[clip,scale=0.4]{Figs/IncBtm.eps}
	\caption{計測波形からFDTD法で再構成した入射速度場}
	\label{fig:inc_btm}
\end{figure}
\begin{figure}[hbt]
\centering
	\includegraphics[clip,scale=0.35]{Figs/experiment2.eps}
	\caption{平板試験体を使った速度波形の計測}
	\label{fig:experiment2}
\end{figure}
このように,計算上は再現されない一部の入射波成分が存在するが,
実際の探触子位置($S$)は斜角入射された横波がプレート下面で一回反射した後,
切り欠き角部近傍に到達するよう設定している.従って,現実の送信源である$S$から
スリットに向けて,プレート下面を経由せずに到達する実体波は弱く,散乱波の発生にほとんど寄与しない.
このことから,スリットへの直達波が再構成された入射場に含まれないことは,
%
実体波を使った
%
今回のRTMイメージングでは問題とならない.
以上により,送信探触子のモデル化を行うことなくRTMイメージングに用いる入射場を
得ることができる.
%%%%%%%%%%%%%%%%%%%%%%%%%%%%%%%%%%%%%%%%%
\subsection{イメージング結果}
{\bf 図}-\ref{fig:inc_btm}の入射場と{\bf 図}-\ref{fig:snap_crack}の
時間反転場から求めた$K_\rho(\fat{x}),K_{\lambda}(\fat{x})$および
$K_{\mu}(\fat{x})$を{図-}\ref{fig:imgs}に示す. 
この図には,3種類のRTMイメージがそれぞれの最大値で無次元化して表示されている.
ただし,散乱源の存在は画像化関数が正の値を取る箇所として示されるため,
画素値の表示範囲は0から1としている.
%\begin{figure}[tbh]
%\centering
%	\includegraphics[clip,scale=0.55]{Figs/imgs.eps}
%	\caption{計測波形を用いて行った3種類のRTMイメージングの結果}
%	\label{fig:imgs}
%\end{figure}
白の実線はFDTDモデル,すなわち,試験体の外形を示し,
紫の実線は後に示す{\bf 図}-\ref{fig:imgs_zoom}において拡大表示する
範囲を表している.なお,RTMイメージングは{\bf 図}-\ref{fig:fd_model}
の解析領域全体で行っているが,{図-}\ref{fig:imgs}に示す継手周辺以外の場所では,
画像化関数の値が小さく,有意な指示が現れないことを確認している.
\\
\hspace{\parindent}
これら3種類のRTMイメージのうち$K_\rho(\fat{x})$と$K_{\mu}(\fat{x})$では,
切り欠き側面とスリット開口部付近に大きな画素値をもつ箇所がある.
{\bf 図}-\ref{fig:imgs_zoom}は,その状況をより詳しく観察するために
{\bf 図}-\ref{fig:imgs}で紫の実線で囲った箇所を拡大したものである.
これら拡大図ではイメージング結果とスリット位置を対照するため,
スリット位置を白の破線で描き加えてある.
{\bf 図}-\ref{fig:imgs_zoom}の(a)と(c)に示される通り,
$K_\rho$と$K_\mu$では,大きな画素値をもつ箇所が,実際の散乱源である
スリットと切り欠き側面に非常によく一致している.
\begin{figure}[tbh]
\centering
	\includegraphics[clip,scale=0.55]{Figs/imgs.eps}
	\caption{計測波形を用いて行った3種類のRTMイメージングの結果}
	\label{fig:imgs}
	\vspace{-3mm}
\end{figure}
\begin{figure}[tbh]
\centering
	\includegraphics[clip,scale=0.55]{Figs/imgs_zoom.eps}
	\caption{計測波形を用いて行った3種類のRTMイメージングの結果
	(スリット周辺を拡大して表示)}
	\label{fig:imgs_zoom}
	\vspace{-5mm}
\end{figure}
一方,スリットの下側は画像に現れず,これはスリット下端側からの
散乱波が元々観測波形に含まれていないことが原因と考えられる.
実際,{\bf 図}-\ref{fig:inc_btm}において入射場の挙動を見ると,
プレート底面側から来る横波はスリット下端部を逸れ,{\bf 図}-\ref{fig:inc_btm}(f)では
切り欠きの上側で大きな振幅を持つことが確認できる.
このことを考慮すれば,観測波形に含まれる横波散乱波は微弱なものであるにも
関わらず,時間反転場の計算を通じて効果的にイメージングに利用されていると言える.
ただし,継手周辺には,相対画素値が0.2から0.4程度となる箇所がスリット
近傍以外にも多数見られ背景のノイズとなっている.これは,継ぎ手内部の残響に
よるため,完全に抑制することは難しい.しかしながら,送受信位置と観測時間範囲には
最適化の余地が大きく,実験上も圧電センサーでの受信を行えば信号強度は大きく
改善できる.そのため,RTMイメージの信号−雑音比を向上させることは今後十分に期待がもてる.
\\
\hspace{\parindent}
$K_\rho(\fat{x}), K_\mu(\fat{x})$が良好な結果を与えるのに対し,
$K_\lambda(\fat{x})$は,スリットの有無を判断できるほどの結果が
現れていない.これは,今回検出された散乱波がスリットで生じた横波散乱波
であることが理由と考えられる.なぜなら,$K_{\lambda}(\fat{x})$はFWIの
観点ではラメ定数$\lambda$に対する目的関数の感度を意味し,
$\lambda$は縦波速度に関係する弾性係数だからである.従って,
スリットで発生した縦波散乱波が観測波形に含まれない限り,$K_\lambda(\fat{x})$
が高い感度を示さないことは必然的な結果と言える.
%\\\hspace{\parindent}
%これまでに示した$K_\rho(\fat{x})$や$K_\mu(\fat{x})$のような超音波イメージを,波形の遅延−重ね合わせに基づく方法で
%得ることは以下の理由から困難と考えられる.第一に,遅延時間の設定には散乱波の経路特定が必要だが,
%時間反転場の計算を行うことなく,今回のように複数回の反射とモード変換を伴うエコー経路を特定することは難しい.
%第二に,経路が正しく特定されとしても,対応するエコー波形をそのまま画像化領域に投影する方法では,エコー
%継続時間がそのまま像の滲みになって現れる. 今回の計算で言えば,{\bf 図}-\ref{fig:snap_crack}(f)
%において赤色で示された波群が占める領域程度の滲みは避けられない.第三に,画像解像度の向上にデコンボリューションなどの
%パルス圧縮技術を援用することも簡単ではない.なぜなら,デコンボリューションには対象波形と類似したスペクトル成分
%を持つ参照波形が必要とされるが,そのような参照波形を経路枚に用意することは現実的ではない.以上より,時間反転法は,
%複雑形状部位の超音波イメージングにおいて,従来法を超える適用範囲と性能が期待できるという点で有望なアプローチと考えられる.


\section{まとめ}
本研究では,実験で得た超音波波形を使った時間反転場の集束挙動解析と,
RTM(reverse time migration)に基づく欠陥の超音波イメージングを行った.
その結果,観測波形中の微弱なエコーを,
散乱源であるスリットに時間反転場で正しく集束させられることを示した.
また,超音波イメージングでは,FWI(fullwave inversion)の${\rm Fr\acute{e}chet}$ Kernelにあたる画像化関数を
用いて欠陥画像を合成し,スリットと切り欠きの中で,入射超音波に照射される部位が
極めて正確に検出されることを示した.これらのことから,時間反転場を利用したイメージング法が,
溶接継手のような複雑な形状をした狭い領域の超音波イメージングにも有用であることを明らかにした.
なお,時間反転場の計算結果として今回検出された散乱波の伝播経路は,溶接ビード表面を経由し,
さらに縦波から横波へのモード変換を伴うものであった.このような経路を辿るエコーの
存在を予め想定し,トモグラフィー的な方法でイメージングを行うことは難しく,
少なくともこの点において,時間反転法には従来の超音波イメージング法にない
明確なメリットがある.
%入射波や周辺境界からの反射波や回折波に比べ,
%微弱なものであることに加え,溶接部を経由どるエコーの存在を
%さらに,時間反転場を使ったRTMイメージングで得られた結果は,
%背景のノイズが多いということで改善の余地はあるものの,
%スリットとその起点となる切り欠き側面を検出しており,同等の正確さでの画像化は
%従来法である開口合成法やその類似法では経路特定と波形変化の予想が難しいという
%意味で原理的に達成困難なものである.以上の結果は,きず周辺での回折や反射に超音波の反響が無視できず,
%エコー伝播経路も複雑となり得る状況で,時間反転集束に基づく
%エコー識別や画像化が,有効なアプローチとなることを示唆する.
%\\
%\hspace{\parindent}
今後は,送受信位置が時間反転場の挙動に与える影響を調べて最適化し,RTMイメージングの
信号-雑音比を改善することが課題となる.また,超音波探傷試験への実装に向け,
時間反転場の計算効率を改善することも重要となる.
%また,実際の探傷には,環境振動や対象物表面の性状に影響を受けやすいレーザー振動計は適さず,
%圧電センサーの利用が望ましい.そのため,時間反転集束法に適したアレイ探触子の開発や使用も,
%高品質な観測データを得るために今後取り組むべき課題となる.
%
%``\verb|\usepackage{jjsce-macros}|''
%を有効にしている場合,図表の参照に
%``\verb|\figref{}|'',
%``\verb|\tabref{}|''
%が使えます.
%\section{参考文献の引用とリスト}
%参考文献は出現順に番号を振り,その引用箇所でこのように\cite{a}上付き右括弧付き数字で指示します.
%\verb|\usepackage[superscript]{cite}|を有効にしている場合,
%``\verb|\cite{a,b,c,d}|''と記述すると,
%\cite{a,b,c,d} と記載されます.
%``文献\verb|\Cite{a}|''と記述すると,
%文献\Cite{a} と上付き添字ではなく文章中に記載されます.
\Acknowledgment % 謝辞:
本研究の一部は,一般社団法人中国建設弘済会,技術開発支援事業(助成期間:2021-2022年度)
の支援を受けて行ったものである.ここに記して謝意を表する.
%\appendix
%\section*{付録の見出し}
%付録のsectionが一つしかない場合は,付録に番号をつける必要がありません.
%``\verb|\section*{}|''を用いてください.
%付録のsectionが複数ある場合は,
%``\verb|\section{}|''を用いてください.
\begin{thebibliography}{9}
% PA
\bibitem{Wilcox2007}
	Wilcox,P. D., Holmes, C., and Drinkwater, B. W.:
	Advanced reflector characterization with ultrasonic phased arrays in NDE applications,
	\textit{ IEEE Trans. Ultraso. Feroelect. Freq. Contr.}, Vol.54, No. 8, pp.1541-1550, 2007.
% SAFT
\bibitem{Thomson1984}
	Thomson, R. N.:
	Transverse and longitudinal resolution of the synthetic aperture focusing technique,
	\textit{Ultrasonics}.,Vol. 22, pp.9-15, 1984. 
\bibitem{Doctor1986}
	Doctor, S. R., Hall, T. E. and Reid, L. D.:
	SAFT -the evolution of a signal processing technology for ultrasonic testing,
	\textit{ NDT International},Vol.19, No.3, pp.163-167, 1986. 
\bibitem{Langenberg1986}
	Langenberg,K.J., Berger,M., Kreutter,Th., Mayer,K., and Schmitz, V.:
	Synthetic aperture focusing technique signal processing,
	\textit{NDT International}, Vol.19, No.3, pp.177-189, 1986.
\bibitem{Schmitz2000}
	Schmitz, V.,  Chakhlov, S. and M\"{u}ller, W.:
	Experiences with synthetic aperture focusing technique in the field,
	\textit{ Ultrasonics}, Vol.38, pp.731-738, 2000.
\bibitem{Spies2012}
	Spies, M., Rieder, H., Dillh\"{o}fer, A.,  Shmitz, V. and M\"{u}ller, W.:
	Synthetic aperture focusing and time-of-flight diffraction 
	ultrasonic imaging -past and present,
	\textit{ J. Nondestruct. Eval. },Vol.31, pp.310-323, 2012.
% TFM
\bibitem{Holmes2008}
	Holmes,C., Drinkwater, B. W., and  Wilcox, P. D.:
	Adcanced post-processing for scanned ultrasonic arrays: Application to 
	defect detection and clasification in non-destructive evaluation,
	\textit{ Ultrasonics},Vol. 48,pp.636-642,2008.
\bibitem{Zhang2010}
	J. Zhang, B. W. Drinkwater , P. D. Wilcox, and A. J. Hunter,
	"Defect detection using ultrasonic arrays: The multi-mode total focusing method",
	\textit{ NDT \& E International}, Vol.43, pp.123-133, 2010.
% POFFIS
\bibitem{Langenberg1989}
	Langenberg,K. J.:
	Introduction to the special issue on inverse problems,
	\textit{ Wave Motion}, Vol. 11, pp.99-112, 1989.
\bibitem{Kitahara2002}
	Kitahara, M., Nakahata, K., and Hirose, S.:
	Elastodynamic inversion for shape reconstruction and type classification of flaws,
	\textit{ Wave Motion},Vol. 36, pp.443-455, 2002.
\bibitem{Shlivinski2007}
	Shlivinski, A. and Langenberg, K. J.:
	Defect imaging with elastic waves in inhomogeneous-anisotropic materials with 
	composite geometries,
	\textit{ Ultrasonics},Vol. 46, pp.89-104, 2007.
% U-rib 
%\bibitem{FDTD_book}
%	佐藤雅弘:FDTD法による弾性振動・波動の解析入門,pp. 34-49,森北出版,2003. 
%	[Satoh, M.: \textit{FDTD-hou niyoru Danseishindou\& Hadou-no-kaiseki-nyumon}, pp. 34-49, Morikita Syuppan, 2003.]
%\bibitem{Urib1}
%	村越 潤, 高橋 実, 小池 光裕, 木村 友則: 臨界屈折角近傍に調整した超音波斜角探触子による鋼床版デッキ進展き裂の探傷法の検討,
%	土木学会論文集A1(構造・地震工学), 68巻, 2号, pp.453-464, 2012.
%	[Murakoshi, J. Takahashi, M., Koike, M., and Kimura, T.: \textit{Study on practical ultrasonic inspection method for fatigue cracks in 
%	steel orthotropic deckplates}, Journal of Japan Society of Civil Engineers, Ser. A1 (Structural Engineering \& Earthquake Engineering 
%	pp.453-464, Vol.68, No.2, 2012.]
%\bibitem{Urib2}
%	八木 尚人, 鈴木 俊光, 若林 登, 村野 益巳, 三木 千壽: 鋼床版トラフリブ溶接部の内在疲労き裂に対するフェーズドアレイ超音波探傷試験の適用,
%	土木学会論文集A1(構造・地震工学), 72巻, 3号, pp.393-406, 2016.  [Yagi, N., Suzuki,T. Wakabayashi, N., Murano, M., and Miki, C.: 
%	\textit{Detection of inner fatigue crack in welding between steel deck plate and through rib by phased array non-destructive testing}, 
%	Journal of Japan Society of Civil Engineers, Ser. A1 (Structural Engineering \& Earthquake Engineering pp.393-406, Vol.73, No.3, 2016.]
\bibitem{Urib3}
	服部 雅史, 牧田 通, 舘石 和雄, 判治 剛, 清水 優, 八木 尚人:
	鋼床版Uリブ・デッキプレート溶接部の内在き裂に対するフェーズドアレイ超音波探傷の測定精度,
	土木学会論文集A1(構造・地震工学),74巻, 3号, pp.516-530, 2018.
	[Hattori,M. Makita,T., Tateishi,K. Hanji,H., Shimizu, M., and Yagi, N.: 
	\textit{Crack sizing accuracy of a phased array ultrasonic scanner developed for 
	inspection of rib-to-deck welded joints in orthotropic steel decks}, 
	Journal of Japan Society of Civil Engineers, 
	Ser. A1 (Structural Engineering \& Earthquake Engineering 
	pp.393-406, Vol.73, No.3, 2016.]
%DORT
\bibitem{FinkTextBook}
	Fink,M., Kuperman, W.A.,  Montagner,J.-P., and Tourin, A.(Eds.):
	Imaging of complex media with acoustic and seismic waves, Springer (2002)
\bibitem{Fink1992}
	Fink,M.:
	Time reversal of ultrasonic fiels - Part I: Basic principles",
	\textit{IEEE Trans. Ultrason., Ferroelect., Freq. Contr.}, Vol.39, pp.555-566, 1992.
\bibitem{Prada1994}
	Prada, C. and Fink, M.: 
	Eigenmode of the time reversal operator: A solution to selective focusing in multiple-target media, 
	\textit{ Wave Motion}, Vol.20, pp.151-163, 1994. 
\bibitem{Prada1995}
	Prada, C. and Fink, M.: 
	Selective focusing through inhomogeneous media: the DORT method,
	\textit{ IEEE Ultrasonics Symposium}, pp.1449-1453, 1995. 
\bibitem{Prada1995_2}
	Prada, C. and Fink, M.: 
	The iterative time reversal process: Analysis of the convergence, 
	\textit{J. Acoust. Soc. Amer.}, Vol.97, pp.62-71, 1995.
\bibitem{Mordant1999}
	Mordant,N., Prada,C., and Fink, M.:
	Highly resolved detection and selective focusing in a waveguide using the D.O.R.T. method,
	\textit{ J. Acoust. Soc. Am.}, Vol.105, No.5, pp.2634-2642, 1999. 
\bibitem{Tanter2000}
	Tanter, M., Thomas, J.-L., and Fink,M.:
	Time reversal and the iterative filter, 
	\textit{ J. Acoust. Soc. Am.}, Vol.108, No.1, pp.223-234, 2000. 
%RTM
\bibitem{Yan2008}
	 Yan, J. and Sava, P.:
	Isotropic angle-domain elastic reverse-time migration,
	\textit{ Geophysics}, Vol.73, No.6, pp.S229-S239, 2008. 
\bibitem{Etgen2009}
	Etgen,J., Gray,S.H., and Zhang, Y.: 
	An overview of depth imaging in exploration geophysics,
	\textit{ Geophysics}, Vol.74, No.6, WCA5-WCA17, 2009.
\bibitem{Velichko2010}
	Velichko, A.  and Wilcox, P. D.:
	An analytical comparison of ultrasonic array imaging algorithm,
	\textit{ J. Acoust. Soc. Am.}, Vol.127, pp.2377-2384, 2010.
\bibitem{Chung2012}
	Chung, W., Pyun,S., Bae,H.S., Shin,C. S., and Marfurt, K.J.:
	Implementation of elastic reverse-time migration using wavefield separation in the frequency domain,
	\textit{Geophys. J. int.}, Vol.189, pp.p1611-1625, 2012. 
\bibitem{Jones}
	Jones, I. F.:
	Tutorial: migration imaging conditions, 
	\textit{first break}, Vol.32, pp.45-55, 2014.
\bibitem{KK_RTM}
	Kimoto,K.,  Nakahata,K., and Saitoh, T.: 
	An elastodynamic computational time-reversal method for shape reconstruction 
	of traction-free scatterers, \textit{Wave Motion}, Vol.72,pp.23-40,2017.
% FWI
\bibitem{Talantola1984}
	Talantola, A.:
	Inversion of seismic reflection data in the acoustic approximation, 
	\textit{Geophysics}, Vol.49, pp.1259-1266, 1984
\bibitem{Fichtner}
	Fichtner, A.: Full seismic waveform modelling and inversion,
	Springer, pp.141-161, 2011. 
% TDTG
\bibitem{Dominguez2005}
	Dominguez,N. and Gbiat, V.:
	Time domain topological gradient and time reversal analogy: 
	an inverse method for ultrasonic target detection, 
	\textit{ Wave Motion}, Vol. 42, No.1, pp.31-52, 2005.
\bibitem{Dominguez2010}
	Dominguez,N. and Gbiat, V.:
	Non-destructive imaging using the time domain topological energy method,
	\textit{ Ultrasonics}, Vol.50, pp.367-372, 2010.
\bibitem{Gibiat2010}
	Gibiat, V. and Shuguet, P.:
	Wave guide imaging through time domain topological energy,
	\textit{ Ultrasonics}, Vol.50, pp.172-179, 2010.
\bibitem{Bonnet2008}
	Bonnet, M.: 
	Topological sensitivity for 3D elastodynamci and acoustic inverse scattering in the time domain,
	\textit{Computer Methods in Applied Mechanics and Engineering},Elsevier,195, pp.5239-5254.2006.
\bibitem{Saitoh2021}
	斎藤 隆泰, 田代 匡彦, 木本 和志: 
	トポロジー感度を欠陥検出指標に用いた3 次元動弾性時間反転解析による欠陥形状再構成, 
	土木学会論文集A2(応用力学), 77巻, 2号, pp.I\_69-I\_79, 2021.
	[Saitoh, T., Tashiro,M.,and Kimoto,K.:
	\textit{Shape reconstruction for a defect using 3-D elastodynamic time-reversal analysis with topologial sensitivity}, 
	Journal of Japan Society of Civil Engineers, Ser.A2 (Applied Mechanics (AM)), 
	Vol.77, No.2, pp.I\_69-I\_79,2021.] 
\bibitem{Nakahata2019}
	中畑 和之, 天野 裕維, 溝田 裕久, 斎藤 隆泰, 木本 和志:
	Wavefield データを利用した数値モデルの構築と時間反転イメージングへの応用,
	土木学会論文集A2(応用力学), 75巻, 2号, pp.I\_71-I\_81, 2019.
	[Nakahata, K., Amano, Y., Mizota, H., Saitoh, T., and Kimoto, K.:
	\textit{Construction of numerical model using wavefield data and its 
	application of flaw imaging by time reversal method},
	Journal of Japan Society of Civil Engineers, Ser.A2 (Applied Mechanics (AM)), 
	Vol.75, No.2, pp.I\_71-I\_81,2019.] 
\bibitem{Schmerr1999}
	Schmerr,L. W.:
	Fundamental of ultrasonic nondestructive evaluation: a modeling approach", Academic Press,1999.	
%FDTD
\bibitem{Fellinger1995}
	Fellinger, P, Marklein,R., Langenberg,K.J., and Klaholz, S.: 
	Numerical modeling of elastic wave propagation and scattering with EFIT 
	-Elastodynamic finite integration technique, 
	\textit{ Wave Motion}, Vol.21, pp.47-66, 1995.
%\bibitem{FDTD_book}
%	佐藤雅弘:FDTD法による弾性振動・波動の解析入門,pp. 34-49,森北出版,2003. 
%	[Satoh, M.: \textit{FDTD-hou niyoru Danseishindou\& Hadou-no-kaiseki-nyumon}, pp. 34-49, Morikita Syuppan, 2003.]
\bibitem{FDTD_KK}
	Kimoto, K. and Ichikawa, Y.:
	A finite difference method for elastic wave scattering by a planar crack with contacting faces
	, \textit{Wave Motion}., Vol. 52, pp.120-137, 2015.
\end{thebibliography}
\end{document}
\bibitem{Drinkwater2006}
	Drinkwater, B. W. and . D. Wilcox, 
	"Ultrasonic arrays for non-destructive evaluation: A review",
	{\em NDT \& E International}, {\bf 39}(2006),525-541. 
\bibitem{a}
	本間仁,安芸皓一:物部水理学,pp. 430-463,岩波書店,1962. 
	[Honma, S. and Aki, K.: \textit{Mononobe Suiri-gaku}, pp. 430-463, Iwanami Shoten, 1962.]
\bibitem{b}
	日本道路協会:道路橋示方書・同解説IV下部構造編,pp. 110-119,1996. 
	[Japan Road Association: \textit{Dorokyo-shihosyo \& Doukaisetsu} IV Kabukouzo-hen, pp. 110-119, 1996.]
\bibitem{c}
	Shepard, F. P. and Inman, D. L.: Nearshore water circu-lation related to bottom topography and wave refraction, 
	\textit{Trans. AGU}., Vol. 31, No. 2, 1950.
\bibitem{d}
	C. R. ワイリー(富久泰明訳):工学数学(上),pp. 123-140,ブレイン図書,1973. 
	[Wylie, C. R. (translated by Tomihisa, Y.): \textit{Advanced Eingineering Mathmatic}, Brain-tosho, 1973.]
%%%%%%%%%%%%%%%%%%%%%%%%%%%%%%%%%%5
\end{thebibliography}
%\input{tables.tex}
%\input{figs.tex}
%\input{caption.tex}
\end{document}
%%
%% End of file `elsarticle-template-1a-num.tex'.
