%%#!platex
%
% Example of Japanese Paper of JSCE
% for LaTeX2e users
%
% revised on 4/25/2014
%
%%%%%%%%%%%%%%%%%%%%%%%%%%%%%%%%%%%%%%%%%%%
%
% もし jis フォントメトリックを使う場合は,以下をアンコメントしてください.
% \DeclareFontShape{JY1}{mc}{m}{n}{<-> s * jis}{}
% \DeclareFontShape{JY1}{gt}{m}{n}{<-> s * jisg}{}
%
\documentclass{jsce}
%
\usepackage{epic,eepic,eepicsup}
%\usepackage{graphicx,multicol}
\usepackage{graphicx}
\usepackage{multicol}
\usepackage{amsmath}
%\usepackage{showkeys}
\usepackage{setspace}
%  amsを使う方は以下をアンコメントしてください.
%\usepackage{amssymb,amsmath}
% 英語はサポートしているかどうか不明
% \inenglish
% 学会サンプルに times とあるので指定しておきます
\usepackage{times}
%
\finalversion
\pagestyle{empty}
%
\title{
	花崗岩の造岩鉱物粒スケールでみた\\
	弾性波伝播特性
}%
\endtitle{
ELASTIC WAVE PROPAGATION IN A GRANITE AT A SCALE OF ROCK FORMING MINERAL GRAINS
}
%
% emailアドレスのフォントをタイプライター体にしたい方は次行をアンコメント
% \emailstyle{\ttfamily}
% emailアドレスを公開される方は,
%% \thanks{○○○○○○\email{your_name@foo.ac.jp}}のようにしてください.
%
\author{木本 和志\thanks{正会員 博士(工学) 
岡山大学 学術研究院環境生命科学学域
(〒700-8530 岡山県岡山市北区3丁目1番地1号)\email{kimoto@okayama-u.ac.jp} (Corresponding Author)}・
岡野 蒼\thanks{学生会員 岡山大学環境生命科学研究科 (〒700-8530 岡山県岡山市北区3丁目1番地1号)}・
斎藤 隆泰\thanks{正会員 博士(工学)群馬大学大学院理工学府 環境創生部門(〒376-8515 群馬県桐生市天神町 1-5-1)}
%佐藤 忠信\thanks{正会員 博士(工学)神戸学院大学・現代社会学部・防災社会学科(〒650-8586神戸市中央区港島1-1-3)}・
%松井 裕哉\thanks{正会員 修士(工学)日本原子力研究開発機構・幌延深地層研究センター・堆積岩処分技術開発Gr
%(〒098-3224 北海道天塩郡幌延町北進432番地2)}
}
\endauthor{Kazushi KIMOTO, Aoi OKANO, Takahiro SAITOH
%\\ Tadanobu SATO and Hiroya MATSUI 
}
%
\abstract{
\small
本研究は,花崗岩における縦波の局所的な伝播挙動を調べたものである.具体的には,造岩鉱物粒スケールでの位相速度を推定することを目的に,花崗岩供試体の板厚方向に透過する縦波をレーザードップラー振動計で計測した.位相速度の評価は別途取得した参照波形と透過波形の位相差に基づいて行い,得られた位相速度を造岩鉱物種毎に統計的に処理して頻度分布を求めた.その結果,カリ長石の位相速度は非対称かつ幅の広い分布に従う一方,石英の位相速度は対称かつより狭い分布に従うことが分かった.さらに,以上で得た頻度分布を用いて花崗岩の2次元数値解析モデルを作成して平面波伝播解析を行ったところ,ガウス分布によって摂動された位相速度分布を与えた解析モデルとは異なる,特異な散乱減衰の挙動が現れることが示された.
}
%
\keywords{granite, rock-forming mineral, ultrasonic wave, phase velocity, probability density}
%
\endabstract{% Yes blank line
\normalsize
This study investigates the propagation characteristics of P-wave in granite. Specifically, the local phase velocity is evaluated experimentally using a granite plate. In the experiment, ultrasonic P-waves transmitted in the thickness direction are measured by scanning the plate surface with a laser Doppler vibrometer. The phase velocity is evaluated based on the phase delay relative to an independently measured reference signal. The phase velocity data obtained thus are processed statistically to produce probability density functions (PDF) for each rock-forming mineral species. As a result, a highly asymmetric and broad PDF is obtained for the P-wave velocity of potassium feldspar, while the PDF for quartz is found to be symmetric and much narrower. Finally, the PDFs are used to generate 2-dimensional numerical models of granite on which plane wave propagation analyses are performed. The numerical results showed that the phase velocity field perturbed according to the experimentally obtained PDFs gives a peculiar decay profile compared to the Gaussian perturbed models.
}
%
% \titlepagecontrol{1}
%
%\receivedate{2019.7.19}
% \receivedate{January 15, 1991}
%
% \def\theenumi{\alph{enumi}}  % もし enumerate 最初の箇条を (a) と
% \def\labelenumi{(\theenumi)} % したい場合・・・
%
\begin{document}
\maketitle
%%%%%%%%%%%%%%%%%%%%%%%%%%%%%%%%%%%%%%%%%%%%%%%%%%%%%%%%%%%%%%%%%%%%%%%%%%%%%
\section{はじめに}
	花崗岩はき裂性の結晶質岩で,新鮮な状態でも造岩鉱物粒程度のスケールでは多数のマイクロクラックを有することが知られている\cite{Kudo1}$^-$\cite{Takagi}.
マイクロクラックは花崗岩の透水性や物質輸送特性,剛性,強度,物理・化学的な風化の進行に影響を与える.
そのため,マイクロクラックの大きさや量を把握することは,例えば,高レベル放射性廃棄物の地層処分のように,
長期に渡って安全性を担保する必要のある処分場を花崗岩体中に建設する際に重要となる.
%
岩石中のき裂の大きさや量,分布状況は偏光顕微鏡で岩石薄片を観察することで調べられる.
しかしながら,岩石薄片の作成には手間がかかる上,岩石内部の状態やき裂の経時的な変化を観察することはできない.
一方,岩石の弾性波試験は結果の定量的解釈が難しいことが多い点に難があるものの,試料内部の状態を簡単な検査で調べることができ,
経時的なモニタリングにも適用できるという面で利点がある.
実際,弾性波の伝播はき裂の影響を強く受けるため,弾性波計測結果からき裂の量や進展に関する情報を得る
試みが行われている\cite{Griffiths}.
ただし,花崗岩を始めとする岩石中の弾性波伝播は鉱物粒やき裂による多重散乱のために非常に複雑で\cite{Sato},
弾性波計測データから鉱物粒径やき裂を定量的に評価する方法は確立されていない.
%
花崗岩中の弾性波伝播特性を理解するためには,多数の鉱物粒から成る多結晶質体モデルを用いて
弾性波の多重散乱解析を行うことが有効となる\cite{FEM}.
そのような多重散乱解析の実施には造岩鉱物の密度と弾性係数値を与える必要がある.
これらの物性値のうち密度は,物質の結晶構造と構成原子の質量から鉱物種ごとに
定めることができ,鉱物種間での差も小さい.
各種造岩鉱物の弾性係数も鉱物学分野で古くから調べられており,文献値\cite{AGU}や
分子シミュレーション\cite{Kawamura}の結果から知ることができる.ただし,それらはき裂の無い
単結晶鉱物に対するものであるため,岩石を構成する鉱物粒の弾性係数としてそのまま用いることはできない.
また,弾性係数は結晶軸方向やき裂の有無による不確実性があるため,対象とする岩石の状態を反映して
確率的に与える必要もある.

花崗岩の弾性波速度や弾性係数に関する研究は,これまでに多数行われている.
例えば,工藤ら\cite{Kudo2},佐野ら\cite{Sano1}は花崗岩コア試料の超音波透過試験を行い,
花崗岩が直交異方性体として振る舞うことを示し,マイクロクラックとの関係を検討している.
ただし,それらの研究では岩石コアスケールでのマクロな異方性や弾性波速度が論じられており,
鉱物粒スケールでの弾性係数や弾性波速度を調べたものではない.
これに対してSivaji\cite{Sivaji}およびFukushima\cite{Fukushima}らは,
レーザードップラー振動計を使って花崗岩試料表面に励起された
超音波振動を可視化し,鉱物粒径と弾性波到達時間のゆらぎの関係を調べている.
岩石のランダムな不均質性に起因した弾性波到達時間のゆらぎについてはRythov近似に基づく方法で
理論的な検討が加えられ,到達時間のゆらぎと不均質性の空間スケールの関係が得られることが示されている
\cite{Muller}$^-$\cite{Baig}.
ただし,これらの研究において、弾性波速度の鉱物粒スケールでの変動は,ガウス分布や指数分布
等、取扱が容易なよく知られた確率分布に従うことが仮定されており,造岩鉱物粒の弾性波速度が従う
実際の確率分布はこれまで調べられていない.

そこで本研究では,超音波計測によって局所的な縦波(P波)の位相速度を統計的に求めることを試みる.
これを多数の鉱物粒について鉱物種毎に行うことで,P波位相速度が従う確率分布を評価する.
以下では,この目的のために行った超音波計測方法について述べた後,実験で得られたP波位相速度分布
の結果を報告する.具体的には,1$\sim$2MHzの周波数帯域におけるP波位相速度の頻度分布を示し,
結晶粒のスケールでは鉱物種に依らず速度分散はほとどんど無いこと,
長石と石英ではP波位相速度の分布が有意に異なることを示す.
また,これら実験で得られたP波位相速度を用いた数値波動解析を行い,縦波平面波の散乱減衰挙動
について検討を行う.その結果,P波位相速度の頻度分布を単一の正規分布と仮定した場合と,
計測から得た頻度分布を用いた場合で散乱減衰の挙動が異なることを示す.
以上の結果から,弾性波媒体として花崗岩をモデル化する上で,実際に計測を行って評価した
位相速度の分布を反映したモデルを用いることが必要であることを述べる.
なお,以下ではP波位相速度を単に位相速度と呼ぶ.


\section{超音波計測実験}
%	%\section{超音波計測実験}
本節では,花崗岩の局所的な弾性波速度を鉱物種毎に調べることを目的として行った,
超音波計測について述べる.
%実験には,薄板状の花崗岩供試体を用い,板厚方向に透過する縦波超音波を計測した.以下にその詳細を記す.
\subsection{実験供試体}
超音波計測に用いた花崗岩供試体を{\bf 図-\ref{fig:fig1}}に示す.
この供試体は岡山県万成地域の採石場で採取した万成花崗岩を,
厚さ3.42mmの平板に切断加工したものである.供試体表面は岩石カッターで
切断したままの状態で,研磨等による仕上げは行っていないが,
表面に凹凸や目視で認められるような欠けや割れ,明らかな風化が無いことを確認している.
ただし,石目の方向は分かっていない.
{\bf 図-\ref{fig:fig1}}(a)は超音波の送受信を行う供試体の表面の全体を,
(b)はその一部を拡大して鉱物粒の状況を示したものである.
 {\bf 図-\ref{fig:fig1}}(b)に記入したように,供試体を構成する主要鉱物は石英(Qt),
カリ長石(K),斜長石(Pl)と黒雲母(Bt)の4種類で,各々の含有割合は
およそ,カリ長石34\%,斜長石17\%,石英44\%,雲母5\%で,これら鉱物種の平均粒径は
順に1.6,0.9,1.1および0.5mm程度である.
なお,ここに示した鉱物組成は,試料表面のデジタルスキャナ画像上で
ピクセル数を鉱物種毎にカウントして得た概算値で,上記4種類以外の
鉱物が含まれないという意味ではない.
また,平均粒径は,デジタル画像上で50mmの線分6本をランダムに引き,
それぞれの線分を横切る鉱物粒の幅と数をカウントした結果から,
鉱物幅の合計(総延長)/鉱物粒数として評価したものである.
ここでの目的は,板厚方向に透過する縦波超音波を計測し,鉱物種毎の位相速度を求めることにある.
そのためには厚み方向には単一の鉱物種が占めるよう,供試体の厚みを鉱物粒径より小さく
することが望ましい.一方,板厚が小さい場合,直接透過波と多重反射波成分が時間軸上
で近接し両者の分離が困難になる.このことから,超音波計測の点では計測に
用いる波長が利用可能な試験片厚さの限界を定める.以下に述べるように,
本研究では1MHzの超音波センサーを用いる.花崗岩の縦波速度はおよそ5$\sim$6km/s程度で,
1MHzでの波長は5$\sim$6mm程度となる.従って,6mmの波長を持つ透過波1波を時間軸上で反射波
から分離して観測するためには3mm以上の板厚が必要になる.このことに加え,
供試体を切断加工して厚みを一定にできること,供試体が実験中に破損することのない
程度の強度を持つことを条件に考慮し,厚さ3.5mmを供試体の設計寸法とした.
なお,実際に切断加工を行って作成したところ,出来上がり寸法では供試体の厚さは3.42mmであった.
\subsection{超音波計測装置の構成}
実験に用いた超音波計測装置の構成を{\bf 図-\ref{fig:fig2}}に示す.
計測系は超音波を送信するための圧電探触子と花崗岩供試体,3軸ステージ,レーザードップラー振動計(LDV),
オシロスコープ,およびスクウェア−ウェーブパルサーで構成されている.
花崗岩供試体は水平2軸,回転1軸の3軸ステージ上に固定し,LDVによるレーザー照射位置を精確に調整する.
その際,送信探触子は供試体の下面に接触させて固定し,供試体とともに移動させる.
探触子の駆動はスクウェア−ウェーブパルサーを用いて行い,100Vの矩形パルス電圧を印加する.
受信にはLDVを用い,供試体上面における鉛直動を時刻歴波形として計測する.
LDVで受信した信号はオシロスコープに転送し,4,096回の平均化を行った後,デジタル波形としてPCに収録する.
このとき,サンプリング周波数は15MHz,計測時間範囲は100$\mu$sとし,全ての位置での計測を同じ条件で行った.
超音波の送信には線集束型の圧電探触子を用い,供試体内に円筒波を励起した.
ここで用いた探触子は曲率半径26.1mm,投影面積25mm×40mmの圧電素子をくさび状の
ポリスチレンシューに取り付けたもので,シュー先端部の幅と長さはそれぞれ2mmと50mm,
共振周波数は1MHzである\cite{Kimoto}.
\subsection{送受信位置}
{\bf 図-\ref{fig:fig3}}に送信および受信点の配置を示す.
図中の${\cal R}(X)$で示した線は送信センサー直上にとった長さ40mmの観測線を表し,
$X$はその位置($x$座標)を意味する.超音波波形の計測は,${\cal R}(X)$上を
ピッチは0.25mmでスキャンして行い,1観測線あたり161点で透過波波形を取得した.
このような透過波計測を$x$方向に1mm間隔で送信センサーを移動させ,
$10\leq X \leq 59$mmの50の観測線において実施することで,合計8,050の波形を取得した.
各観測点における鉱物種は供試体表面をスキャナで撮影したデジタル画像上で特定し,
鉱物種毎の統計を取ることができるようにする.{\bf 図-\ref{fig:fig4}}は鉱物種の特定を
行った結果を示したもので,{\bf 図-\ref{fig:fig1}}のスキャナ画像に対し,
各画像ピクセルにおいて特定した鉱物種を塗り分けて示したものである.
鉱物種の特定は画像編集ソフト上で各画素の
色彩に基づき手作業で行った.これを供試体の上面(観測側)と下面(送信側)の両側に
ついて行い,表裏面で異なる鉱物種となっている観測点での波形データは除外し,
送信側と観測側で同一鉱物種と判定された位置での波形だけを位相速度の算出に用いることとした.
その結果,黒雲母で6点,石英590点,カリ長石825点,斜長石549点
でのデータが最終的に得られた.なお,黒雲母の含有量は他の造岩鉱物に比べて小さく
得られた波形データ数も6点と非常に少ないことから,
以下では黒雲母を除く3つの鉱物の位相速度についての検討を行う.\\
%
\hspace{\parindent}
なお,以上の方法で波形データを選択した場合も,鉱物の平均粒径は
供試体板厚の半分より小さく,表面と異なる鉱物種が内部に介在する
可能性は残る.しかしながら,超音波の波長より小さな鉱物粒は,
周囲の鉱物粒と相互作用の下で運動や変形をするため,仮に,
任意に小さな鉱物粒の挙動を独立に観測できたとして,その結果から鉱物粒塊
の応答を知ることは簡単でない.従って,花崗岩中の波動伝播波挙動を
実験によって調べるには,波長程度のボリュームをもった領域が超音波に
どのように応答するかを見ることがより直接的である.このことを踏まえれば,
目的とする鉱物種の影響を見えなくしてしまう程顕著で無い限り,
透過波の伝播経路上に異種鉱物が存在することはあまり大きな問題では
ないと考えられる.
%
%%%%%%%%%%%%%%%%%%%%%%%%%%%%%%%%%%%%%%%%%%%%%%%%%%%%%%%%%%%%%%
\begin{figure}
\begin{center}
\includegraphics[clip,scale=0.35]{Figs/samples.eps}
\caption{
	超音波計測に用いた花崗岩供試体.
	K,Pl,Qt,Btは,それぞれカリ長石,斜長石,石英および黒雲母を意味する.
}
\label{fig:fig1}
\end{center}
	\vspace{-5mm}
\end{figure}
\begin{figure}[t]
\begin{center}
\includegraphics[clip,scale=0.45]{Figs/ut_setup.eps}
\caption{ 超音波計測装置の構成. }
\label{fig:fig2}
\end{center}
%	\vspace{-15mm}
\end{figure}
\begin{figure}[t]
\begin{center}
\includegraphics[clip,scale=0.45]{Figs/cod.eps}
\caption{
	超音波の送信位置と観測線${\cal R}(X)$の配置.
}
\label{fig:fig3}
\end{center}
%	\vspace{-15mm}
\end{figure}
\begin{figure}[t]
\begin{center}
\includegraphics[clip,scale=0.5]{Figs/map.eps}
\caption{
	画像ピクセル毎に特定した鉱物種の分布
	(緑:石英,赤:カリ長石,白:斜長石,黒:黒雲母).
}
\label{fig:fig4}
\end{center}
%	\vspace{-15mm}
\end{figure}
%%%%%%%%%%%%%%%%%%%%%%%%%%%%%%%%%%%%%%%%%%%%%%%%%%%%%%%%%%%%%%


\section{計測結果}
%	\input{3_data}
\section{位相速度}
%	\input{4_processing}
\section{非均質媒体モデルによる波動伝播解析}
%	\input{5_fdm}
\section{まとめ}
本研究では,花崗岩の鉱物粒スケールにおける縦波位相速度を,局所的な超音波計測よって推定した.
その結果,実験に用いた花崗岩供試体の主要造岩鉱物である,石英,斜長石,カリ長石の
それぞれについて,位相速度の頻度分布を得た.これらの頻度分布の平均値は鉱物種による差が小さく,
いずれも5.4km/s程度と典型的な花崗岩の位相速度に近い値をとり,周波数1MHzから2MHz程度の帯域
ではほとんど速度分散も無いことが分かった.一方,頻度分布の形状や分布幅は鉱物種によって大きく異なり,
次のような特徴があることが明らかとなった.
\begin{itemize}
\item
石英の位相速度は,平均値周辺に集中した対称な頻度分布を持つ.
\item
カリ長石の位相速度は,標準偏差が0.9km/sと大きく非対称な頻度分布を持つ. 
\item
斜長石の位相速度は,複数のピークをもつ頻度分布を描き,分布幅は
石英とカリ長石の中間程度の値となる.
\end{itemize}
以上の結果を,文献に与えられた単結晶鉱物の弾性係数から計算した位相速度と比較する
ことで,花崗岩中の鉱物粒はマイクロクラックの影響により,位相速度がき裂の無い単結晶
に比べて明らかに小さな値を取ることを確認した.一方,分布幅に関して言えば,
単結晶に比べて花崗岩中の石英は頻度分布の分散が小さく,き裂の存在が異方性を
弱める方向に寄与している可能性があることが分かった.さらに,カリ長石は,単結晶
同様,位相速度の頻度分布幅が広く,き裂の存在下でも強い異方性を示すことを明らか
にすることができた.最後に,これら鉱物種に応じた位相速度の頻度分布を考慮して
波動伝播解析を行ったところ,位相速度が単一のガウス分布に従うと仮定した
場合とは異なる散乱減衰挙動が見られることが分かった.
このことは,花崗岩のような複数の鉱物から構成される非均質媒体を伝わる弾性波の
挙動を詳しく調べるためには,構成鉱物毎に位相速度あるいは弾性係数が従う確率分布を
与えた解析が必要となることを示していると考えられる.
%逆に言えば,花崗岩中を伝播する弾性波
%は鉱物組成に反応すると言え,両者の詳しい関係を知ることができれば,弾性波計測結果から
%粒径分布やき裂の量を推定する非破壊検査法として利用できる可能性がある.
%こういった利用方法を
今後は,供試体の劣化や石目方向による位相速度分布の変化を実験的に調べることや,
横波についても同様な計測と位相速度分布の推定を行うことが課題となる.
また,波動伝播挙動のどのような側面に位相速度の確率分布の特徴が反映されるかを,
数値シミュレーションによって特定することも重要な課題の一つとなる.
以上のことに加え,き裂や結晶粒のサイズと方位の分布を考慮したモデルから位相速度
を求め実測値と比較することで,位相速度分布からき裂に関する情報を得ることが可能
になると考えられる.本研究はそのような取り組みの一環として位置付けられる.
%%%%%%%%%%%%%%%%%%%%%%%%%%%%%%%%%%%%%%%%%%%%%%%%%%%%%%%%%%%%%%%%%%%%%%%%%%%%%
\\

{\gt 謝辞:}
本実験に用いた花崗岩供試体は浮田石材店代表浮田隆司氏に提供頂いた.
また本研究の推進には,科学研究費補助金(基盤研究©課題番号\#18K04334)の補助を受けた.
併せて謝意を表す.
%%%%%%%%%%%%%%%%%%%%%%%%%%%%%%%%%%%%%%%%%%%%%%%%%%%%%%%%%%%%%%%%%%%%%%%%%%%%%
%\newpage
%\lastpagecontrol[2cm]{13.7cm}
\begin{thebibliography}{99}
\begin{spacing}{1.175}
\bibitem{Kudo1}
	工藤洋三, 橋本堅一, 佐野修, 中川浩二: 花崗岩の力学的異方性と岩石組織欠陥の分布,
	土木学会論文集, 第370号/III-5, pp.189-197, 1986.
\bibitem{Kudo1991}
	工藤洋三, 橋本堅一, 佐野修, 中川浩二: 花崗岩内に発生するクラックと鉱物粒の関係,
	資源・素材学会誌, 第107巻,第7号, pp.423-427, 1991.
\bibitem{Chin}
	陳友晴,西山孝,喜多浩之,佐藤稔紀: 微小クラックの分類による稲田花崗岩と栗橋花崗閃緑岩の力学的弱面について,
	応用地質,第38巻,第4号,pp.196-204,1997.
\bibitem{Takagi}
	高木秀雄, 三輪成徳, 横溝佳侑, 西嶋圭, 円城寺守, 水野崇, 天野健治:土岐花崗岩中の石英に発達するマイクロクラックの三次元
	方位分布による古応力場の復元と生成環境, 地質学雑誌, 第114巻, 第7号, pp.321-335, 2008.
\bibitem{Griffiths}
	Griffiths, L., Lengline, O., Heap, M.,J., Baud, P., and Schmittbuhl, J.: Thermal cracking in Westerly granite monitored
	using direct wave velocity, coda wave interferometry and acoustic emissions, {\it JGR Solid Earth}, Vol.123, No.3, pp.2246-2261, 2018.
\bibitem{Sato}
	Sato, H., Fehler, M.C., and Maeda, T.:Seismic wave propagation and scattering in the heterogeneous earth, 
	Springer, 2012.
\bibitem{FEM}
	Pamel, A., V., Sha, G., Rokhlin, S., I., and Lowe, M., J., S.:
	Finite-element modelling of elastic wave propagation and scattering within heterogeneous media, 
	{\it Proc. R. Soc. A}, 473:20160738.
\bibitem{AGU}
	Anderson, O., L., Schreiber, E., Liebermann, R., C., and Soga, H.: 
	Some elastic constant data on minerals relevant to geophysics, 
	{\it Review of Geophysics}, Vol.4, No.4, pp.491-524, 1968.
\bibitem{Kawamura}
	河村雄行: 分子シミュレーション鉱物学のすすめ,
	鉱物学雑誌,第28巻,第4号,pp.151-158,1999.
%\bibitem{Bass}
%	Bass, J., D.: Elasticity of minerals, glasses, and melts, 
%	{\im Mineral Physics and Crystallography:A handbook of physical constants }, Vol.2,pp.45-63,  
\bibitem{Pamel}
	Pamel, A., V., Sha, G., Rokhlin, S.,I., and Lowe, M.,J.,S.: 
	Finite-element modelling of elastic wave propagation and scattering within heterogeneous media, 
	{\it Proc. R. Soc. A}, 473:2016738, 2016.
\bibitem{Kudo2}
	工藤洋三, 橋本堅一, 佐野修, 中川浩二:瀬戸内地方の採石場における花崗岩石の異方性, 
	土木学会論文集, 第382号/III-7, pp.45-53, 1987.
%\newpage
\bibitem{Sano1}
	佐野修, 工藤洋三, 河嶋智, 水田義明:異方性体としての花崗岩の弾性率に関する実験的研究, 
	材料, 第37巻, 第418号, pp.84-90, 1987.
%\bibitem{Sano2}
%	佐野修, 民部雅史, 平野亮, 工藤洋三, 水田義明:弾性的対称性未知の岩石の弾性定数決定に関する研究, 
%	材料, 第40巻, 第449号,pp.96-102, 1990.
%\bibitem{Nishizawa1996}
%	西澤修, 雷興林, 佐藤隆司:不均質媒体での地震波伝モデル実験-レーザードップラー速度計を用いた波動計測-
%	, 地震調査所月報, 第47巻, 第4号, pp.209-222, 1996.
\bibitem{Sivaji}
	Sivaji, C., Nishizawa, O., Kitagawa, G., and Fukushima, Y.:A physical-model study of the statistics of seismic waveform fluctuation in random heterogeneous media, 
	{\it Geophys. J. Int.}, Vol.148, pp.575-595, 2002. 
\bibitem{Fukushima}
	Fukushima, Y., Nishizawa, O., Sato, H., and Ohtake, M.:Laboratory study on scattering characteristics of shear waves 
	in rock samples, {\it Bulltine of Seismological Society of America}, Vol.93, No.1, pp.253-263, 2003.
\bibitem{Muller}
	Muller, G., Roth, M. and Korn, M.:Seismic-wave traveltimes in random media,
	{\it Geophys. J. Int.}, Vol.110, pp.29-41, 1992. 
\bibitem{Korn}
	Korn, M.:Seismic waves in random media, 
	{\it Journal of Applied Geophysics}, Vol.29, pp.247-269, 1993.
\bibitem{Spetzler2001}
	Spetzler, J. and Snieder, R.:The effect of small-scale heterogeneity on the arrival time of waves, 
	{\it Geophys. J. Int.}, Vol.145, pp.786-796, 2001. 
\bibitem{Spetzler}
	Spetzler, J., Sivaji, C., Nishizawa, O., and Fukushima, Y.:A test of ray theory and scattering theory based on
	a laboratory experiment using ultrasonic waves and numerical simulation by finite-difference method, 
	{\it Geophys. J. Int.}, Vol.148, pp.165-178, 2002. 
\lastpagecontrol[0.0cm]{19.0cm}
\newpage
\bibitem{Baig}
	Baig, A., M., and Dahlen, F., A.: Traveltime biases in random media and S-wave discrepancy, 
	{\it Geophys. J. Int.}, Vol.1158 pp.992-938, 2004. 
\bibitem{Kimoto}
	木本和志,岡野蒼,斎藤隆泰,佐藤忠信,松井裕哉: 超音波計測に基づく花崗岩中の表面波伝播特性に関する研究,
	土木学会応用力学論文集A2(応用力学),第76巻,第2号,pp.I\_97-I\_108, 2020.
\bibitem{Rose}
	Rose, J. L.: Ultrasonic waves in solid media, Cambridge university press, 1999. 
%\bibitem{RockPhys}
%	ゲガーン, Y., パルシアウスカス, V.: 
%	岩石物性入門, シュプリンガー・ジャパン, 2008. 
%\bibitem{NishizawaI}
%	西澤修:岩石中の地震波伝播I:不均質媒体のモデル化と弾性波速度, 地学雑誌, 第114巻, 第6号,  pp.921-948,  2005.
%\bibitem{Nishizawa2001}
%	西澤修, 雷興林, チャダラム シバジ:不均質媒質での地震波伝播モデル実験, 
%	地震, 第54巻, pp.171-183, 2001.
%\bibitem{Okubo2012}
%	大久保 慎人, 雑賀 敦, 鈴木 貞臣, 中島 唯貴: 地震動観測による地震波速度と岩石物性試験による弾性波速度の関係
%	-段発発破波形の相関による地震波速度構造推定-: 地震, 第65巻, pp.21-30, 2012.
%\bibitem{Rwk_textbook}
%	Klaffter, J., and Sololov,I.M.,秋元 琢磨(訳): ランダムウォークはじめの一歩,共立出版, 2018.
\end{spacing}
\end{thebibliography}
\begin{flushright}
	\small
	\bf{ (Received June 18, 2021)\\
	(Accepted November 30, 2021)}
\end{flushright}
%\newpage
%\lastpagecontrol[0.0cm]{18.0cm}
%\lastpagecontrol[0.0cm]{13.0cm}
\end{document}

%\lastpagesettings
%\begin{minipage}[c]{13.7cm}
%\end{minipage}
%\lastpagecontrol[0cm]{13.7cm}
%\begin{multicols}{1}
%-------------------------------------------------
%-------------------------------------------------
%\end{multicols}

