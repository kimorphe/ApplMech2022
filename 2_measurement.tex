%\section{超音波計測実験}
本節では,花崗岩の局所的な弾性波速度を鉱物種毎に調べることを目的として行った,
超音波計測について述べる.
%実験には,薄板状の花崗岩供試体を用い,板厚方向に透過する縦波超音波を計測した.以下にその詳細を記す.
\subsection{実験供試体}
超音波計測に用いた花崗岩供試体を{\bf 図-\ref{fig:fig1}}に示す.
この供試体は岡山県万成地域の採石場で採取した万成花崗岩を,
厚さ3.42mmの平板に切断加工したものである.供試体表面は岩石カッターで
切断したままの状態で,研磨等による仕上げは行っていないが,
表面に凹凸や目視で認められるような欠けや割れ,明らかな風化が無いことを確認している.
ただし,石目の方向は分かっていない.
{\bf 図-\ref{fig:fig1}}(a)は超音波の送受信を行う供試体の表面の全体を,
(b)はその一部を拡大して鉱物粒の状況を示したものである.
 {\bf 図-\ref{fig:fig1}}(b)に記入したように,供試体を構成する主要鉱物は石英(Qt),
カリ長石(K),斜長石(Pl)と黒雲母(Bt)の4種類で,各々の含有割合は
およそ,カリ長石34\%,斜長石17\%,石英44\%,雲母5\%で,これら鉱物種の平均粒径は
順に1.6,0.9,1.1および0.5mm程度である.
なお,ここに示した鉱物組成は,試料表面のデジタルスキャナ画像上で
ピクセル数を鉱物種毎にカウントして得た概算値で,上記4種類以外の
鉱物が含まれないという意味ではない.
また,平均粒径は,デジタル画像上で50mmの線分6本をランダムに引き,
それぞれの線分を横切る鉱物粒の幅と数をカウントした結果から,
鉱物幅の合計(総延長)/鉱物粒数として評価したものである.
ここでの目的は,板厚方向に透過する縦波超音波を計測し,鉱物種毎の位相速度を求めることにある.
そのためには厚み方向には単一の鉱物種が占めるよう,供試体の厚みを鉱物粒径より小さく
することが望ましい.一方,板厚が小さい場合,直接透過波と多重反射波成分が時間軸上
で近接し両者の分離が困難になる.このことから,超音波計測の点では計測に
用いる波長が利用可能な試験片厚さの限界を定める.以下に述べるように,
本研究では1MHzの超音波センサーを用いる.花崗岩の縦波速度はおよそ5$\sim$6km/s程度で,
1MHzでの波長は5$\sim$6mm程度となる.従って,6mmの波長を持つ透過波1波を時間軸上で反射波
から分離して観測するためには3mm以上の板厚が必要になる.このことに加え,
供試体を切断加工して厚みを一定にできること,供試体が実験中に破損することのない
程度の強度を持つことを条件に考慮し,厚さ3.5mmを供試体の設計寸法とした.
なお,実際に切断加工を行って作成したところ,出来上がり寸法では供試体の厚さは3.42mmであった.
\subsection{超音波計測装置の構成}
実験に用いた超音波計測装置の構成を{\bf 図-\ref{fig:fig2}}に示す.
計測系は超音波を送信するための圧電探触子と花崗岩供試体,3軸ステージ,レーザードップラー振動計(LDV),
オシロスコープ,およびスクウェア−ウェーブパルサーで構成されている.
花崗岩供試体は水平2軸,回転1軸の3軸ステージ上に固定し,LDVによるレーザー照射位置を精確に調整する.
その際,送信探触子は供試体の下面に接触させて固定し,供試体とともに移動させる.
探触子の駆動はスクウェア−ウェーブパルサーを用いて行い,100Vの矩形パルス電圧を印加する.
受信にはLDVを用い,供試体上面における鉛直動を時刻歴波形として計測する.
LDVで受信した信号はオシロスコープに転送し,4,096回の平均化を行った後,デジタル波形としてPCに収録する.
このとき,サンプリング周波数は15MHz,計測時間範囲は100$\mu$sとし,全ての位置での計測を同じ条件で行った.
超音波の送信には線集束型の圧電探触子を用い,供試体内に円筒波を励起した.
ここで用いた探触子は曲率半径26.1mm,投影面積25mm×40mmの圧電素子をくさび状の
ポリスチレンシューに取り付けたもので,シュー先端部の幅と長さはそれぞれ2mmと50mm,
共振周波数は1MHzである\cite{Kimoto}.
\subsection{送受信位置}
{\bf 図-\ref{fig:fig3}}に送信および受信点の配置を示す.
図中の${\cal R}(X)$で示した線は送信センサー直上にとった長さ40mmの観測線を表し,
$X$はその位置($x$座標)を意味する.超音波波形の計測は,${\cal R}(X)$上を
ピッチは0.25mmでスキャンして行い,1観測線あたり161点で透過波波形を取得した.
このような透過波計測を$x$方向に1mm間隔で送信センサーを移動させ,
$10\leq X \leq 59$mmの50の観測線において実施することで,合計8,050の波形を取得した.
各観測点における鉱物種は供試体表面をスキャナで撮影したデジタル画像上で特定し,
鉱物種毎の統計を取ることができるようにする.{\bf 図-\ref{fig:fig4}}は鉱物種の特定を
行った結果を示したもので,{\bf 図-\ref{fig:fig1}}のスキャナ画像に対し,
各画像ピクセルにおいて特定した鉱物種を塗り分けて示したものである.
鉱物種の特定は画像編集ソフト上で各画素の
色彩に基づき手作業で行った.これを供試体の上面(観測側)と下面(送信側)の両側に
ついて行い,表裏面で異なる鉱物種となっている観測点での波形データは除外し,
送信側と観測側で同一鉱物種と判定された位置での波形だけを位相速度の算出に用いることとした.
その結果,黒雲母で6点,石英590点,カリ長石825点,斜長石549点
でのデータが最終的に得られた.なお,黒雲母の含有量は他の造岩鉱物に比べて小さく
得られた波形データ数も6点と非常に少ないことから,
以下では黒雲母を除く3つの鉱物の位相速度についての検討を行う.\\
%
\hspace{\parindent}
なお,以上の方法で波形データを選択した場合も,鉱物の平均粒径は
供試体板厚の半分より小さく,表面と異なる鉱物種が内部に介在する
可能性は残る.しかしながら,超音波の波長より小さな鉱物粒は,
周囲の鉱物粒と相互作用の下で運動や変形をするため,仮に,
任意に小さな鉱物粒の挙動を独立に観測できたとして,その結果から鉱物粒塊
の応答を知ることは簡単でない.従って,花崗岩中の波動伝播波挙動を
実験によって調べるには,波長程度のボリュームをもった領域が超音波に
どのように応答するかを見ることがより直接的である.このことを踏まえれば,
目的とする鉱物種の影響を見えなくしてしまう程顕著で無い限り,
透過波の伝播経路上に異種鉱物が存在することはあまり大きな問題では
ないと考えられる.
%
%%%%%%%%%%%%%%%%%%%%%%%%%%%%%%%%%%%%%%%%%%%%%%%%%%%%%%%%%%%%%%
\begin{figure}
\begin{center}
\includegraphics[clip,scale=0.35]{Figs/samples.eps}
\caption{
	超音波計測に用いた花崗岩供試体.
	K,Pl,Qt,Btは,それぞれカリ長石,斜長石,石英および黒雲母を意味する.
}
\label{fig:fig1}
\end{center}
	\vspace{-5mm}
\end{figure}
\begin{figure}[t]
\begin{center}
\includegraphics[clip,scale=0.45]{Figs/ut_setup.eps}
\caption{ 超音波計測装置の構成. }
\label{fig:fig2}
\end{center}
%	\vspace{-15mm}
\end{figure}
\begin{figure}[t]
\begin{center}
\includegraphics[clip,scale=0.45]{Figs/cod.eps}
\caption{
	超音波の送信位置と観測線${\cal R}(X)$の配置.
}
\label{fig:fig3}
\end{center}
%	\vspace{-15mm}
\end{figure}
\begin{figure}[t]
\begin{center}
\includegraphics[clip,scale=0.5]{Figs/map.eps}
\caption{
	画像ピクセル毎に特定した鉱物種の分布
	(緑:石英,赤:カリ長石,白:斜長石,黒:黒雲母).
}
\label{fig:fig4}
\end{center}
%	\vspace{-15mm}
\end{figure}
%%%%%%%%%%%%%%%%%%%%%%%%%%%%%%%%%%%%%%%%%%%%%%%%%%%%%%%%%%%%%%

