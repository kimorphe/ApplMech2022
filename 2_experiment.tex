Fig.1に超音波エコー計測の概要を示す.計測にはT溶接継手の形状を模擬した試験体を用いた.Fig.1はその断面図を示したもので紙面奥行き方向には一様な形状をもつ.試験体を構成する水平材(フランジ)の厚さは12mm,鉛直材(ウェブ)の厚さは7.8mmで,継手部分には隅肉溶接での接合を想定し,余盛りとルートギャップを表現した箇所を設けてある.なお,ここでは,継手の形状だけを模擬することを意図し,試験体はアルミニウムブロックを切削加工して作成している.Fig.2は継手近傍の詳細を示したもので,余盛り表面は半径20mmの円弧とし,脚長7mmとなっている.き裂はルートギャップから発生し,フランジ側に進展するケースを考え,長さ4mm,幅0.2mmのスリットを,鉛直方向から傾き15度の方向に放電加工で作成した.なお,き裂の起点はルートギャップを模擬した1mm角のスリット角部とした. 超音波の送受信はウェブ右側,フランジの上面からのみ可能と仮定し,Fig.1のように送受信センサーを配置した.これは,TOFD法を適用することのできない,制約の厳しい探傷条件を想定したものである.ここで,送信には屈折角70度,公称中心周波数5MHzの圧電探触子を,受信にはレーザードップラー振動計(LDV)を用いた.送信探触子は,溶接始端部から40mmの位置に,探触子前縁部が来るように設置した.一方,LDVによる受信は,止端部から0.2mmの間隔で201点,40mmの範囲で行った.その際,サンプリング周波数は80MHz,平均化回数は4,096回とし,送信探触子は振幅300V,幅0.1sの矩形パルスで駆動した.

Fig.3は,このようにして計測した201点での超音波波形を示した走時プロットである.横軸は経過時間(s)を,縦軸は位置 (mm)を表し,対応する時空間点でのエコー強度(mV)をカラーマップで示している.Fig.3において右下がりの軌跡は0方向への,右上がりのものは0方向への進行波を表す.き裂からの後方散乱波は右方向への進行波として観測される.しかしながら,右方向への進行波にはルートギャップや溶接止端部等からの形状エコーが含まれるため,進行方向や到達時間だけから,き裂エコーを識別することは困難である. 
\begin{figure}[htb]
\centering
%\rule{20mm}{20mm}
	\includegraphics[clip,scale=0.5]{Figs/samples.eps}
%\includegraphics[clip,scale=0.5]{Figs/.eps}
\caption{図のキャプションは図の下に置く}
\label{fig:sample}
\end{figure}

