\subsection{試験体}
鋼床版を構成するデッキプレートとUリブは,橋軸方向にみて
{\bf 図}-\ref{fig:spec}(a)のような断面形状をしており,
丸で囲った位置で溶接接合されている.
溶接は閉断面をつくるUリブの外側から隅肉溶接で行われ,
継手形状を模式的に示すと{\bf 図}-\ref{fig:spec}(b)の
ようになっている.
リブ断面の内側にある溶接ルート部には,ルートギャップと呼ばれる
間隙が溶接後に残ることがあり,ここを起点として疲労き裂が生じること
が報告されている.き裂は交通荷重によって生じる応力場に依存して,
溶接ビード内部へ進展する場合や,デッキプレートを貫通する方向に
進展する場合があることが知られている.
{\bf 図}-\ref{fig:spec}(b)は後者のケースを示し,本研究ではこの
タイプのき裂検出を想定した超音波擔傷試験について議論を行う.
{\bf 図}-\ref{fig:spec}(c)はそのために作成した試験体の断面形状を,
(d)は外観を示したものである.
なお,(d)の写真は全体形状がわかり易く示されるように,(c)の写真に対して上下反転
させた状態で撮影されている.以下では,(d)の状態におかれた試験体向きを基準とし,
試験体表面に原点をおく$xyz$直交座標系をこの写真にあるようにとる. 
この試験体は,Uリブとデッキプレートが作るT継手の超音波探傷を,
デッキプレートの下面($y=0$)から行う状況を再現することを意図したものである.
ただし,ここでは溶接や疲労試験ではなく,アルミニウムブロックを切削加工して
試験体を作成した.これは,溶接継手部の形状を模擬し,試験体各部の形状と
寸法が精確に規定された試験体を用いるためである.
以下では試験体のUリブに相当する箇所をリブ,デッキプレートに相当する箇所を
プレートと呼ぶ.各々の板厚は,{\bf 図}-\ref{fig:spec}にあるように,
リブが7.8mm, プレートが12.0mmで,これらは実橋でよくある寸法に合わせてある.
なお,鋼とアルミニウムでは密度は大きく異なるが,弾性波速度の差は少なく,
同じ周波数であれば波長にも大きな差はない.そのため,実橋の寸法をそのままを
反映して試験体の板厚を設定した.
一方,試験体の長さ(180mm),幅(80mm),高さ(36mm)は,後述する精密ステージでの
位置決めに支障が無い程度のサイズと重量であると同時に,現実には存在しない
リブやプレート端部からの反射波と継手部からのエコーが分離できる程度に大きく
することを条件として決定した.
%%%%%%%%%%%%%%%%%%%%%%
\begin{figure}[b]
	%[htb]
\centering
	\includegraphics[clip,scale=0.5]{Figs/samples.eps}
\caption{超音波エコー測定のための試験体}
\label{fig:spec}
\end{figure}
%%%%%%%%%%%%%%%%%%%%%%
リブ-プレート交差部の詳細は{\bf 図}-\ref{fig:model}に示す通りで,
脚長6mm程度の余盛りを想定した半径20mmの扇形状の領域を設けてある.
き裂はルートギャップから発生し,フランジ側に進展するケースを考え,
長さ4mm,幅0.2mmのスリットを,鉛直方向から傾き15度で放電加工によって
作成した.また,き裂の起点となるルートギャップはた1mm角のスリットで
表現し,この角部が模擬き裂であるスリットの起点となっている.
なお,試験体の断面形状は紙面奥行き$z$方向には一様である.
\subsection{超音波計測方法}
{\bf 図}-\ref{fig:experiment}に超音波エコー計測の概要を示す.
超音波の送受信はリブからみて右側に位置するプレート表面$(y=0)$からのみ
送受可能と仮定し,{\bf 図}-\ref{fig:experiment}のようにセンサーを配置した.
実際のUリブ継手の探傷では,リブ外面にもセンサーを置くことは可能だが,
ここではより制約の厳しい探傷条件に相当する送受条件を採用する.
超音波の送信には屈折角70度,公称中心周波数5MHzの横波(SV波)圧電探触子を,
受信にはレーザードップラー振動計(Laser Doppler vibrometer:LDV)を用いた.
送信探触子は,溶接始端部に相当する$x=0$mmから40mmの位置に探触子前縁部が来るように設置した.
一方,LDVによる受信は,止端部から0.2mmの間隔で201点,40mmの範囲で行った.
LDVでの受信は,サンプリング周波数80MHz,平均化回数は4,096回とし,送信探触子の駆動は
超音波パルサー/レシーバを用い,振幅300V,幅0.1$\mu$sの矩形パルスで行った.
なお,送信探触子の位置は,入射波が底面で一回反射した後にスリット起点である
切り欠きの角部に到達する位置としている.
LDVを受信に用いた理由は,非接触法で波動場を乱すことなく観測できること,
点レシーバとみなすことができること,アレイ探触子と異なり,測定点間隔を
任意に設定できることの三点である.
LDVと試験体の相対位置の調整と測定点位置の移動は,試験体を3軸ステージ
(水平2軸,水平面内回転1軸)に固定し,試験体位置を精確に移動させることによって行った.
%%%%%%%%%%%%%%%%%%%%
\begin{figure}[htb]
\centering
	\includegraphics[clip,scale=0.55]{Figs/model_bead.eps}
	\caption{試験体断面の形状と寸法}
	\label{fig:model}
\end{figure}
%%%%%%%%%%%%%%%%%%%%
\begin{figure}[htb]
\centering
	\includegraphics[clip,scale=0.35]{Figs/experiment.eps}
	\caption{超音波の送受信位置および方法}
	\label{fig:experiment}
\end{figure}
%%%%%%%%%%%%%%%%%%%%
\subsection{計測波形}
{\bf 図}-\ref{fig:bscans}は,このようにして計測した201点での超音波波形を示した走時プロットである.
横軸は経過時間($\mu$s)を,縦軸は位置(mm)を表し,対応する時空間点でのエコー強度(mV)をカラーマップで示している.
(a)は{\bf 図}-\ref{fig:spec}に示したスリットがある試験体での計測結果を,(b)はスリットが無い試験体での計測結果である.
スリットが無い試験体は,{\bf 図}-\ref{fig:spec}のものとは別途用意したもので,模擬き裂としてのスリットが
無いこと以外,寸法,形状ともスリットがあるものと同じにしてある.
{\bf 図}-\ref{fig:bscans}において左上がりの軌跡は左方向への,右上がりのものは右方向への進行波を表す.
き裂からの後方散乱波は右方向への進行波として観測される.しかしながら,右方向への進行波にはルートギャップや
溶接止端部等からの形状エコーが含まれるため,進行方向や到達時間だけから,スリットからのエコーを識別することは困難である. 
(a)と(b)の結果を比較しても,スリットの有無による結果の違いは明確でなく,スリットからのエコーを
走時プロット上で同定することは難しい.スリットの有無による違いは,理論上は両者の結果の差分を取ることで得られる.
しかしながら,実際の計測では,計測点位置やセンサーの接触状況,LDV計測では試験体表面の状況も
正確に一致していない限り,これらの条件の差が計測波形にも現れ,スリット以外からの波形を十分な
精度で相殺することは困難である つまり,{\bf 図}-\ref{fig:bscans}のように比較対象がある場合でも,スリットからのエコーを検出することが
困難な状況にあるといえ,この結果からエコー伝播経路や伝播モードを特定することも難しい問題であることが分かる.
このことは,従来の画像化方法である開口合成法やフェーズドアレイ法の適用が難しいことも意味する.
勿論,あらかじめエコー伝播経路を仮定して像合成を行うことは可能ではある.しかしながら,その際に仮定した
経路やモードが正しいものであったかを判断するには,再現実験やシミュレーションに頼らざるを得ず,
これらの方法のみ自律して信頼性を担保してくれるものではない.
%%%%%%%%%%%%%%%%%%%%
\begin{figure}[htb]
\centering
	\includegraphics[clip,scale=0.45]{Figs/bscans.eps}
	\caption{計測波形の走時プロット}
	\label{fig:bscans}
\end{figure}

