%%%%%%%%%%%%%%%%%%%%%%
\begin{figure}[thb]
	%[htb]
\centering
	\includegraphics[clip,scale=0.5]{Figs/samples.eps}
\caption{超音波エコー測定のためのアルミニウム試験体}
\label{fig:spec}
\end{figure}
%%%%%%%%%%%%%%%%%%%%%%
%%%%%%%%%%%%%%%%%%%%
\begin{figure}[htb]
\centering
	\includegraphics[clip,scale=0.55]{Figs/model_bead.eps}
	\caption{アルミニウム試験体断面の形状と寸法}
	\label{fig:model}
\end{figure}
%%%%%%%%%%%%%%%%%%%%
\subsection{試験体}
鋼床版を構成するデッキプレートとUリブは,橋軸方向にみて{\bf 図}-\ref{fig:spec}(a)
のような断面を有し,丸で囲った箇所は溶接で接合されている.溶接は閉断面Uリブの外側から
隅肉溶接で行われ,継手部分は{\bf 図}-\ref{fig:spec}(b)のようになっている.
リブ断面内側の溶接ルート部にはルートギャップと呼ばれる間隙が溶接後に残ることがあり,
この位置を起点とした疲労き裂の発生が報告されている\cite{Urib3}.
き裂は交通荷重による応力場の状況に応じ,溶接ビード内部へ進展する場合や,
デッキプレートを貫通する方向に進展する場合があることが知られている.
{\bf 図}-\ref{fig:spec}(b)は後者のケースを示し,本研究ではこの
タイプのき裂検出を想定した超音波探傷試験の室内実験とイメージングを行う.
{\bf 図}-\ref{fig:spec}(c)はそのために作成した試験体の断面形状を,
(d)は試験体全体の様子を示したものである. (d)の写真では,全体形状が把握
できるように(c)の写真に対して上下反転させた状態で試験体が置かれている.
以下では,(d)の状態で定置された試験体向きを基準とし, 試験体表面に原点をもつ
$xyz$直交座標系をこの写真のように定める. 
%
本試験体は,Uリブとデッキプレートが作るT溶接継手の超音波探傷を,
デッキプレート下面($y=0$)から行う状況を想定したものである.
ただし,ここでは溶接継手の形状だけを模擬し,試験体各部の形状と寸法が精確に
規定された試験体を用いるために,溶接や疲労試験は行わず,アルミニウムブロックを
切削加工して試験体を作成している.従って,正確には"溶接継手"や"溶接ビード"
などの呼称で試験体各部を指すことはできないが,"継手を模擬した形状”や,
"ビードに相当する箇所"といった表現上の煩わしさを避けるため,アルミニウム
試験体各部を対応する溶接継手部位の名称で呼ぶ.
また,Uリブにあたる箇所をリブ,デッキプレート部分をプレートと呼ぶ.
これらの板厚は,{\bf 図}-\ref{fig:spec}にあるように,
リブが7.8mm, プレートが12.0mmで,実橋によくある寸法に合わせてある.
一方,試験体全体のサイズは,計測位置を調整するための精密ステージに
設置する上で支障が無いよう,長さ(180mm),幅(80mm),高さ(36mm)とした.
なお,鋼とアルミニウムで密度は大きく異なるが,弾性波速度の差は小さい.
従って,同じ周波数であれば波長にも大差はないため,試験体材料をアルミニウム
とすることによる試験体寸法の換算や調整は行っていない.
リブ-プレート継手部の詳細は{\bf 図}-\ref{fig:model}に示すようであり,
脚長6mmの余盛りを想定した半径20mmの扇形の領域が設けられている.
き裂はルートギャップから発生し,プレート貫通方向に進展する
ケースに対応させ,長さ4mm,幅0.2mmのスリットを,鉛直方向から傾き15度で
放電加工を行って作成した.また,き裂の起点となるルートギャップは,1mm角の
切り欠きで表現し,この角部がスリットの起点(開口部)となる.
なお,試験体の断面形状は紙面奥行き($z$軸)方向に一様である.
\subsection{超音波計測方法}
{\bf 図}-\ref{fig:experiment}に超音波エコー計測の概要を示す.
超音波の送受信は,リブからみて右側のプレート表面$(y=0)$でのみ可能と仮定し,
{\bf 図}-\ref{fig:experiment}のようにセンサーを配置した.
実際のUリブ継手の探傷では,リブ外面にもセンサーを置くことは可能だが,
ここではより制約の厳しい探傷条件を設定している.
%%%%%%%%%%%%%%%%%%%%
\begin{figure}[thb]
\centering
	\includegraphics[clip,scale=0.35]{Figs/experiment.eps}
	\caption{超音波計測系の構成と送受信位置}
	\label{fig:experiment}
\end{figure}
%%%%%%%%%%%%%%%%%%%%
超音波の送信には屈折角70度,公称中心周波数5MHzの横波(SV波)圧電探触子を,
受信にはレーザードップラー振動計(Laser Doppler Vibrometer:LDV)を用いた.
%
なお,送信に用いた斜角探触子の振動子は10mm$\times$40mmの矩形で,{\bf 図}-\ref{fig:experiment}
の紙面奥行き($z$軸)方向に40mmの幅がある.つまり,送信源の幅は5MHzでの横波波長の約63倍と
十分に大きい.これは,超音波の送信軸($y$軸)付近では2次元的な波動場を励起することを意図して
設計したものである.
%
送信探触子は,溶接止端部に位置する座標原点から40mm離れた位置に探触子前縁部
が来るように設置した.これにより,入射波は底面で一回反射した後,
概ね切り欠き角部に到達することになる.
一方,LDVによる受信は,止端部から開始して0.2mm間隔,
201点,40mmの範囲(図中$R$)で行った.
%LDV受信時のサンプリング周波数は80MHz,平均化回数は4,096回とした.
送信探触子の駆動には超音波パルサーを用い,
振幅300V,幅0.1$\mu$sの矩形パルスを印加した.
%
送信パルスの繰返し周波数は1kHz,LDVによる観測はサンプリング周波数80MHzで行い,
4096回の計測結果をスタッキングして各観測点で一つの時刻歴波形を得た.
%
受信にLDVを用いた理由は,非接触法で波動場を乱すことなく観測できること,
アレイ探触子と異なり,測定点間隔を任意に設定できることの2点である.
試験体とLDVの相対位置調整は,測定点位置の移動(スキャンニング)を含め,
3軸ステージ(水平2軸,水平面内回転1軸)のテーブル上に試験体を載せ,
試験体を移動させることによって行った.
\subsection{計測波形}
{\bf 図}-\ref{fig:bscans}は,以上の方法で計測した201の超音波波形を示した走時プロット
である.横軸は経過時間($\mu$s)を,縦軸は観測位置(mm)を表し,対応する時空間点でのエコー
強度(mV)をカラーマップで示している.
%
なお,LDV計測ではレーザー照射方向の振動速度に比例した電圧波形が得られるため,
エコー強度は試験体表面の法線方向速度とみなして差し支えない.
%
(a)は{\bf 図}-\ref{fig:spec}に示したスリットがある試験体での
計測結果を,(b)はスリットが無い試験体での計測結果である.スリットが無い試験体は,
{\bf 図}-\ref{fig:spec}のものとは別に用意したもので,スリットを設けない点を除き寸法や形状は
同一である.
%%%%%%%%%%%%%%%%%%%%
\begin{figure}[bht]
\centering
	\includegraphics[clip,scale=0.45]{Figs/bscans.eps}
	\caption{計測波形の走時プロット.(a)スリットあり,(b)スリットなしの場合.}
	\label{fig:bscans}
\end{figure}
これらの走時波形において左上がりの軌跡は左($x<0$)方向への,
右上がりのものは右($x>0$)方向への進行波を表す.従って,探触子からの入射波は
左上がり,き裂からの後方散乱波は右上がりの軌跡を描く.ただし,右進行波には
ルートギャップや溶接止端部等からの形状エコーも含まれるため,進行方向や
到達時間だけを基準にスリットからのエコーを識別することは難しい.
%また,理論上は,スリットからのエコー成分は2つの走時波形(a)と(b)の差分として得られる.
%
また,計測誤差が無視できる場合には,走時波形(a)と(b)の差分をとることで,スリットに
起因する信号成分を抽出することができる.
%
しかしながら,実際の計測では,計測点位置やセンサーの接触状況,試験体表面の状況が
互いに正確に一致していない限り,スリットに関与しない波形成分を十分な精度で
相殺することはできない.つまり,{\bf 図}-\ref{fig:bscans}のように比較対象がある場合でも,
複数のエコーや反射波が混在する今回のような状況では,観測結果からエコー伝播経路や
伝播モードを特定することは簡単でない.このことは,経路やモードが想定できることを前提とする
開口合成法やフェーズドアレイ法の適用が難しいことも意味する.


