時間反転集束法では観測波形を時間反転して媒体へ入力し,
時間を遡る方向へ散乱波の伝播を調べる.
これを実験や数値シミュレーションで行う場合,時間反転波形をどのような物理量として
入力するか考える必要がある.ここでは,前節で示した観測波形を時間反転し,
表面力として数値解析モデルに与える.これは,フルウェーブインバージョン(fullwave inversion: FWI)
における随伴場を計算することに相当する.そこで本節では動弾性問題の基礎方程式と,
FWIにおける随伴問題の定式化をはじめに示す. 次に,観測波形を境界値に用いた随伴問題の解析結果を示し,
散乱源への波動場の集束状況を調べる.数値波動解析は全て二次元(面内波)問題として行い,ベクトルやテンソル
成分は総和規約を適用した表現を用いる. 以下では,随伴問題の解である随伴場を,統一して時間反転場と
呼ぶことにする.
\subsection{動弾性問題の基礎式}
{\rm 図}-\ref{fig:fd_model}に示すような,試験体形状に合わせた領域$G$,
時間範囲$T=(0,\, t_f)$における超音波の伝播・散乱問題を,線形弾性体における
動弾性問題として考える.ここで,位置ベクトルを$\boldsymbol{x}=(x_1,x_2)$,時間変数を$t$とし,
物体力は働かないものとする.座標系は原点位置を含め,実験と一致させ,
$(x,y)=(x_1,x_2)$であるとする.いま,応力テンソルを$\sigma_{ij}$, 速度ベクトルを$v_i$,
質量密度を$\rho$とすれば,運動方程式は
\begin{equation}
	\rho \dot{v}_i=\sigma_{ji,j}, \ \ (\boldsymbol{x},t)\in G\times T
	\label{eqn:}
\end{equation}
と表される.ただし$(,)$は空間の,$\dot{()}$は時間の微分
\begin{equation}
	(\cdot)_{,i}=\frac{\partial (\cdot)}{\partial x_i}, \ \ 
	\dot{(\cdot)}=\frac{\partial (\cdot)}{\partial x_i}, \ \ 
	\label{eqn:}
\end{equation}
を表す.速度$v_i$は変位ベクトル$u_i$の時間微分で与えられ,
ひずみテンソル$\varepsilon_{ij}$は,変位を使って
\begin{equation}
	\varepsilon_{ij}=\frac{1}{2}(u_{i,j}+u_{j,i})
	\label{eqn:FWD}
\end{equation}
と表される.弾性係数テンソルを$C_{ijkl}$とすれば,フック則は
\begin{equation}
	\sigma_{ij}=C_{ijkl}\varepsilon_{kl}
	\label{eqn:}
\end{equation}
と表され,等方性体ではラメ定数$\lambda$と$\mu$を用いて
\begin{equation}
	C_{ijkl}=\lambda \delta_{ij}\delta_{kl} +
	\mu (
	\delta_{ik}\delta_{jl}
	+
	\delta_{il}\delta_{jk} 
	)
	\label{eqn:Cijkl_iso}
\end{equation}
とすることができる.このとき,縦波および横波の位相速度は,それぞれ,
\begin{equation}
	c_{L}=\sqrt{\frac{\lambda + 2\mu}{\rho}}
	, \ \ 
	c_{T}=\sqrt{\frac{\mu}{\rho}}
	\label{eqn:}
\end{equation}
となる.$t=0$を超音波の送信時刻とすれば,初期条件は
\begin{equation}
	u_i(\boldsymbol{x},0)=0, \ \ v_i(\boldsymbol{x},0)=0, \ \ \boldsymbol{x} \in G
	\label{eqn:IC}
\end{equation}
としてよい.超音波の励起は領域境界$\partial G$の一部$S$に加えた
表面力$\bar{t}_i^{(n)}$によって行われ,それ以外の領域境界で
表面力働かないとする.このとき境界条件は次のように表される.
\begin{equation}
	t_i^{(n)} = \sigma_{ji}n_j=
	\left\{
		\begin{array}{ll}	
			\bar{t}^{(n)}_i, & (\fat{x},t)\in S\times T \\
			0,  & (\boldsymbol{x},t)\in (\partial G \setminus S)\times T) 
		\end{array}
	\right.
	\label{eqn:BC}
\end{equation}
ただし$n_i$は外向きの単位法線ベクトルを表す.
\\
\hspace{\parindent}
以上の定式化に関する3点の留意事項を述べる.
第一に,解析領域$G$は継手周辺部分の試験体断面を型どったもので,
試験体全体をモデル化したものではない.
第二に,数値解析モデルにはスリットは含まれない.
これは,スリットの無いモデルでの計算結果からスリットの検出
や画像化を行うことがここでの目的となるためである.
最後に,領域$S$に加えられる表面力$\bar{t}_i^{(n)}$は,
物理的には超音波探触子と試験体の接触力によって生ずるもので,
直接測定はできない.そこで,超音波から入射される波動場を
再現するために$\bar{t}^{(n)}_i$を与えた計算を行うのではなく,
プレット裏面側の振動波形を使って入射場を再構成する.
この点については,第4節においてあらためて説明する.
%%%%%%%%%%%%%%%%%%%%%%%%%%%%%%%%%%%%%%%%%%%%
\subsection{随伴問題(時間反転場の支配方程式系)}
超音波探傷で検出すべき欠陥を,弾性係数や密度が周囲と異なる領域(インクルージョン)と
みなせば$C_{ijkl}$や$\rho$の分布を推定することで欠陥を検出ができる.FWIは
これらのモデルパラメータ:
\begin{equation}
	\fat{m}=\left( \rho, \, C_{ijkl}\right)
	\label{eqn:mprms}
\end{equation}
を推定するために利用することのできる方法の一つである.
FWIでは,観測波形と仮定したモデルパラメータ$\fat{m}$に対して得られる
シミュレーション波形の差を最小化するよう$\fat{m}$を推定する.
いま,未知の欠陥が領域$G$に含まれるとして,境界$\partial G$の一部
$R\in\partial G$で観測を行い,次のデータを得たとする.
\begin{equation}
	\left\{ 
	\left. 
	u_n^{meas}(\boldsymbol{x},t)\right|  (\boldsymbol{X},t)\in R\times T
	\right\}
	\label{eqn:data}
\end{equation}
ここで,$u_n=u_in_i$は変位の法線方向成分を表す.
いま,前項で述べた波動伝播問題の解として得られたシミュレーション波形を
$u(\boldsymbol{x},t)$とし,FWIにおける目的関数を
\begin{equation}
	\chi(\fat{m}):= \frac{1}{2} \int _T\int _R \left|u_n-u_n^{meas}\right|^2 dtds
	\label{eqn:cost}
\end{equation}
とする.$\chi(\fat{m})$の次の意味での$\delta \fat{m}=(\delta \rho,\, \delta C_{ijkl})$
方向への微分
\begin{equation}
	\nabla \chi (\fat{m})\delta \fat{m} = \lim_{\varepsilon \rightarrow 0}
	\frac{1}{\varepsilon}
	\left\{
		\chi(\fat{m}+\varepsilon \delta \fat{m})
		-
		\chi(\fat{m})
	\right\}, 
	\label{eqn:del_chi}
\end{equation}
は,
\begin{equation}
	\nabla \chi(\fat{m}) = 
	-\int_G K_{\rho}(\fat{x})\delta \rho dv
	- 
	\int_G K_{ijkl}(\fat{x}) \delta C_{ijkl} dv
	\label{eqn:}
\end{equation}
の形で表される.$K_\rho(\fat{x})$と$K_{ijkl}(\fat{x})$は
Fr$\acute{\rm e}$chet kernelと呼ばれ,
随伴問題の解$u^{\dagger}(\fat{x},t)$を用いて次のように与えられることが知られている.
\begin{eqnarray}
	K_\rho(\fat{x}) &= & \int _T \dot{u}_i \dot{u}^{\dagger}_i dt 
	\label{eqn:K_rho}
	\\
	K_{ijkl}(\fat{x}) &= & \int _T u_{i,j} u^{\dagger}_{k,l} dt 
	\label{eqn:K_ijkl}
\end{eqnarray}
ここで,$u_i^{\dagger}$を解にもつ随伴問題は,次のような終端値-境界値問題として
定式化される.
\begin{itemize}
\item
支配方程式:
\begin{equation}
	\rho \ddot{u}_i^{\dagger} =
	\sigma^{\dagger}_{ji,j}, \ \ 
	(\fat{x},t) \in G\times T
	\label{eqn:wveq_adj}
\end{equation}
ただし,
\begin{equation}
	\sigma_{ij}^{\dagger}= C_{ijkl}u_{k,l}^{\dagger}.
	\label{eqn:sigma_dgg}
\end{equation}
\item: 
終端条件:
\begin{equation}
	u_i^{\dagger}(\fat{x},t_f) =0,  \ \
	\dot{u}_i^{\dagger}(\fat{x},t_f) =0,  \ \, \fat{x}\in G
	\label{eqn:IC_adj}
\end{equation}
\item:
境界条件:
\begin{equation}
	t_i^{(n)\dagger}
	=
	\sigma_{ji}^{\dagger}n_j =
	\left\{
		\begin{array}{ll}
			-(u_n-u_n^{meas})n_i, & \fat{x}\in R \\
			0, & \fat{x} \in \partial G\setminus R
		\end{array}
	\right.
	\label{eqn:BC_adj}
\end{equation}
\end{itemize}
式(\ref{eqn:wveq_adj})-(\ref{eqn:BC_adj})で表される随伴問題は,
\begin{equation}
	\tau=t_f-t
	\label{eqn:tau_def}
\end{equation}
の時間反転操作によって$\tau$を新しい時間変数とすることで,弾性波の
初期値-境界値問題に帰着される.勿論,物理的な時間$t$に関して言えば時間を
遡る方向に問題を解くことになるため,その結果得られる随伴場
$u_i^{\dagger}(\fat{x},\tau)$は時間反転場とみなすことができる.
\begin{figure}[htb]
\centering
	\includegraphics[clip,scale=0.55]{Figs/sim_model.eps}
	\caption{解析対象領域.各部の寸法はスリットが無いことを除き
	試験体({\rm 図}-\ref{fig:model})と同じ.}
	\label{fig:fd_model}
\end{figure}
\subsection{時間反転場の数値解析法と計算条件}
時間反転場の数値解析,すなわち,随伴問題(\ref{eqn:wveq_adj})-(\ref{eqn:BC_adj})の解析
には2次元FDTD法を用いる.解析領域は図-\ref{fig:fd_model}とし,実験における
観測領域である$R$に,時間反転した波形を表面力として加える.
ただし,領域の打ち切り位置で反射波発生することを避けるために,図-\ref{fig:fd_model}の
プレートとリブの打ち切り位置には15mmの厚さのPML吸収領域を設けておく.
モデルパラメータであるラメ定数$\lambda, \mu$と密度$\rho$は,実測したアルミニウム試験体
の位相速度
\begin{equation}
	c_L=6.35{\rm km/s}, \ \ c_T=3.15{\rm km/s}
	\label{eqn:phase_vels}
\end{equation}
と,密度の文献値$\rho=2.7{\rm g/cm}^3$を与えた.
FDTD法における空間と時間の離散化幅$h$と$\Delta t$は,それぞれ,
\begin{equation}
	h=0.05{\rm mm}, \ \ \Delta t=3{\rm ns}
	\label{eqn:}
\end{equation}
とし,解析時間範囲は$T=(0,45)\mu$s,空間範囲は幅90mm高さ35mmの矩形
領域とし,媒体が存在しない空間領域も含めスタガード格子を配置した.
その結果,時間ステップ数は$15,000$,空間格子数は未知量あたり
1,800$\times$700とした.\\
\hspace{\parindent}
随伴問題の境界条件(\ref{eqn:BC_adj})では,$u_n^{meas}$として実験で得られた
超音波波形を用いる.LDV計測では速度波形が与えられるが,
%FDTD法では速度と応力を未知数として解析を行う.その際,境界条件も数値解析上は速度境界条件
%として与えるため,LDVをによる観測波形をそのまま用いることができる.
順問題の解である$u_n(\fat{x},t)$には,入射波変位$u_n^{in}$を与えることが望ましい.
そうすることができれば,$u_n^{meas}$は全変位だから$u_n-u_n^{meas}$は散乱波成分を与え,
散乱源に戻る波動場を明瞭に観察できる.しかしながら,実測波形にはノイズが含まれること,
本来は3次元問題を計算コストを抑えるために2次元問題として扱っていることから,
実測波形に含まれる入射波成分を正確に差し引くことは難しい.
とりわけLDVをによる計測では,試料表面の状態に応じた受光感度の変化があり,
その補正も必要となり更に困難が多い.そこで本研究では,観測波形から別途用意した
入射波成分を引くのではなく,スリットから受信点側に伝わる右($x>0$)方向への
進行波成分を散乱波成分とみなし時間反転場の計算に用いる.
すなわち,観測波形$u_n^{meas}(x,t)$を,
右進行波$u_n^{+}(x,t)$と左進行波$u_n^{-}(x,t)$の和:
\begin{equation}
	u_n^{meas}(x,t)=u_n^{+}(x,t)+u_n^{-}(x,t)
	\label{eqn:split}
\end{equation}
で表し,$u_n^-=-(u_n^+-u_n^{meas})$を時間反転場解析の境界値として用いる.
式(\ref{eqn:split})の分離は,観測波形を2次元フーリエ変換して波数-周波数スペクトル:
\begin{equation}
	U(k,\omega)= \frac{1}{(2\pi)^2} \iint u^{meas}_n(x,t)e^{-ikx+i\omega t}dtdx
	\label{eqn:Ukw}
\end{equation}
を計算し,$k$と$\omega$同符号のスペクトル成分は$u_n^-$に,
異符号の成分は$u_n^+$に寄与すると考え,帯域制限した次のフーリエ逆変換で求める.
\begin{eqnarray}
	u_n^-(x,t) &=& \iint H(-k\omega)U(k,\omega)e^{ikx-i\omega t}dk d\omega
	\label{eqn:um} \\
	u_n^+(x,t) &=& \iint H(k\omega) U(k,\omega)e^{ikx-i\omega t}dk d\omega 
	\label{eqn:up}
\end{eqnarray}
ここで,$H(s)$はヘビサイドステップ関数を表す.
{\rm 図}-\ref{fig:kwfilted_xt}の(a)と(b)は,{\rm 図}-\ref{fig:bscans}(a)に示した
データを$\dot u_n^{meas}(x,t)$として,式(\ref{eqn:Ukw})-(\ref{eqn:up})で計算した
左右の進行波速度波形$\dot{u}^-,\dot{u}^+$を走時プロットとして示すものである.
なお,これらの速度波形は$\dot u_n^{meas}(x,t)$の最大値で規格化したものである.
%計測位置をあらわすための
%$(x,y,z)$座標系と波動問題の定式化で用いた位置ベクトルの関係は
%$\fat{x}=(x_1,x_2)=(x,y)$である.
{\rm 図-}\ref{fig:kwfilted_xt}(a)の結果では,右上がりの波群はほとんど見られず,
左進行波$\dot u_n^-$が適切に取り出されていることが分かる.
一方,{\rm 図}-\ref{fig:kwfilted_xt}(b)では,主として右進行波$\dot u_n^{+}$から
なることが確認できるが,一部,振幅の大きな左進行波が除去しきれいてない.
これは,LDV観測における反射強度のばらつきにより,速度振幅値に不規則な空間変動
が加わることにに起因したものと考えられる.
\begin{figure}[htb]
\centering
	\includegraphics[clip,scale=0.5]{Figs/kwfilted_xt.eps}
	\caption{観測波形の走時プロット((a)左進行波成分,(b)右進行波成分,(c)表面波成分を除去した右進行波)}
	\label{fig:kwfilted_xt}
\end{figure}
図\ref{fig:kwfil}
%%%%%%%%%%%%%%%%%%%%%%%%%%%%%%%%%%%%%%%%%%%%
\subsection{時間反転場の挙動}
式(\ref{eqn:BC_adj})において$u_n(x,t)=u_n^-(x,t)$とし,
\begin{equation}
	u_n(x,t) - u_n^{meas}(x,t)
	=
	u_n^-(x,t) - u_n^{meas}(x,t)
	=
	u_n^+(x,t)
	\label{eqn:BC_rwv}
\end{equation}
を境界値としてFDTD法で計算した時間反転場を{\rm 図}-\ref{fig:snap_crack_rwv}に示す.
この図は,速度場$| \dot{\fat{u}}_n^{\dagger}(\fat{x},t)|$のスナップショットを6つの
異なる時刻$\tau$について示したものである.
ただし,粒子速度の値は,適度なコントラストで波動場が可視化されるようにスケールした相対値である.
$\tau$は実時間$t$と式(\ref{eqn:tau_def})の関係にあるため,{\rm 図}-\ref{fig:snap_crack_rwv}
は(a)から(f)の順に物理的な時間を遡る方向への進展挙動を示している.
(a)の時刻では,大きな振幅を持った箇所は表面近傍に集中している.
これは,表面波が大きな振幅をもって伝播していることを示すものである.
少し時間が経過した(b)では,溶接止端部である$x_1=0$の位置で回折波が生じ,
継手内部へ主としてSV波として進展する.このようにして回折によって
発生したSV波は必然的に空間的に広がり集束しない.
従って,本来の散乱源であるスリットの検出や位置の特定にはあまり有効ではない.
\begin{figure}
%	[htb]
\centering
	\includegraphics[clip,scale=0.4]{Figs/SnapCrackRwv.eps}
	\caption{時間反転場の進展状況(スリットがある場合)}
	\label{fig:snap_crack_rwv}
\end{figure}
そこで,表面は成分を入力波形$u_n^{-}(x_1,t)$から取り除いて,時間反転場の計算を行う.
表面波の位相速度$c_R$は,等方弾性体の場合$c_L$と$c_T$から理論的に求められる.
また,計測波形から求めた右進行波$u_n^+(x_1,t)$の波数-周波数
スペクトル$U^+(k,\omega)=H(-k\omega)U(k,\omega)$上で,傾き$c_R^{-1}$の傾きの
直線に沿って分布する.{\rm 図}-\ref{fig:kwfilted_kw}(a)はこのことを示す
ために$|U^+(k,\omega)|$をプロットしたものである.
横軸は周波数$f=\frac{\omega}{2\pi}$,縦軸は波数$\xi=\frac{k}{2\pi}$を表し,
スペクトル振幅をカラー表示している.
この中で、傾き$1/c_R=1/3.05$(km/s)$^{-1}$の原点を通る直線付近の成分が
表面波で,非常に狭い範囲に分布していることが分かる.そこで,この直線に
そってガウス型の窓関数を作用させ,表面波成分を取り除いたのが
{\rm 図}-\ref{fig:kwfilted_kw}(b)の結果である.これを,時空間領域に
フーリエ逆変換すると{\rm 図}-\ref{fig:kffilted_xt}(c)の走時波形が得られ,
同図(b)に見られた右あがりの直線的な指示が概ね取り除かれていることが分かる.
これを境界値として用い計算した時間反転場を{\rm 図}-\ref{fig:snap_crack}に示す.
\begin{figure}[htb]
\centering
	\includegraphics[clip,scale=0.4]{Figs/kwfilted_kw.eps}
	\caption{計測波形の波数-周波数スペクトル}
	\label{fig:kwfilted_kw}
\end{figure}
この図には,時間反転波形の入力域$R$から当初下向きに進んだ波が,プレート下面で
反射されてビード部分に向かう様子が(a)と(b)に現れている.
続く(c)-(e)では,ビード表面で反射されて向きを変えた波が,
スリットに向かって集束していく様子が示されている.
また、ここに見える波群の進行速度から,励起された直後は縦波,ビード表面で反射
された後は横波として伝わっていると考えられる.
このような経路と伝播モードが十分検出しうる強度で観測波形に含まれることを
事前に知ることは困難で、表面波の除去と時間反転場の計算によってはじめて明らかと
なったと言える.経路の把握や選択が予め可能でない以上,開口合成法のような方法で
ここで見出されたような散乱波を像合成に利用することはできず,本来波形に
含まれる散乱波の情報を十分に活用することができない.

同じようにして,スリットが無い試験体で得られた波形から計算した時間反転場を
{\rm 図}-\ref{fig:snap_none}に示す.ここでも,観測波形({\rm 図}-\ref{fig:bscans}(b))
を進行方向で分離した後,表面波成分の除去を行い,その時間反転波形をFDTD解析に
用いて,時間反転場を計算している.
そのためデータ処理上の差は全く無く,観測波形自体を比較してもあまりはっきりとした
違いは無い.しかしながら,{\rm 図}-\ref{fig:snap_none}に示し結果は,スリットが
ある場合と大きく異なり,横波がモード変換することなく,プレート下面で一回反射した
後,ルートギャップを模擬した切り欠きのコーナ部分に集束していることが分かる.
両者の違いは,スリットの有無のみであるため、互いを比較することでスリットが存在する
ことを知ることができることが分かる.
このことは,観測波形を適切に処理すれば,今回のように多数の反射波や回折波
が存在する反響環境下でもき裂検出と位置推定が可能であることを意味している.
従来のイメージング法では散乱波の経路を予め把握しておく必要があり,
多数の経路が想定される今回のような条件では適用が困難である.一方,時間反転法では,
散乱源に集束する波動場を得るにあたり,伝播経路や伝播モードに関する仮定や事前情報
を必要とせず,その点に大きなメリットがある.{\rm 図}-\ref{fig:snap_crack},
{\rm 図}-\ref{fig:snap_none}はこのことを例示する良好な結果と言える.
\begin{figure}[htb]
\centering
	\includegraphics[clip,scale=0.4]{Figs/SnapCrack.eps}
	\caption{時間反転場の進展状況(スリットがある場合)}
	\label{fig:snap_crack}
\end{figure}
\begin{figure}[htb]
\centering
	\includegraphics[clip,scale=0.4]{Figs/SnapNone.eps}
	\caption{時間反転場の進展状況(スリットがない場合)}
	\label{fig:snap_none}
\end{figure}
