線形弾性波の支配方程式は時間反転操作に関して普遍である.そのため,観測で得られた試験体表面の振動波形を境界値として初期値-境界値問題を時間に関して逆方向に解くことで,媒体内部の散乱波伝播挙動を調べることができる.き裂からの散乱波が観測波形に含まれる場合,時間反転波形で励起された波動場(時間反転場)の一部は,散乱波の発生源であるき裂に向かって集束すると期待される.その様子を観察できれば,き裂位置や向き等の推定につながる有用な情報を観測波形から抽出できる.ただし,観測波形は試験体表面の一部でしか得られず,時間反転問題における初期条件も正確には知りえない.さらに,観測ノイズも存在することから,時間反転場の再構成は完全ではありえない.従って,再構成された時間反転場において,散乱源に集束する波動場が必ずしも得られるとは限らない.
時間反転場の計算には2次元FDTD法を用いた.その際,  Fig.3に示した速度波形を,によって時間反転し,を新たな時間変数としたを境界値に用いた.ただし,sで,における試験体内部の場は未知であるため,時間反転場解析における初期速度と初期応力場はともゼロと仮定した.媒体は非減衰,等方均質な線形弾性体とし,縦波および横波の位相速度を,それぞれ実測値から6.35km/sと3.15km/sとした.FDTD法による離散化は,これらの位相速度に対し安定条件を満足するよう,空間格子間隔を0.05mm, 時間ステップ間隔を0.0125sとした.
Fig.4は以上の方法で得られた時間反転場のsにおけるスナップショットである. (a)がき裂を含む供試体に対する結果を,(b)がき裂の無い供試体に対する結果で,粒子速度の大きさをカラーマップで表示している.ただし,粒子速度の値は,適度なコントラストで波動場が可視化されるようにスケールした相対値である.これらの結果から明らかなように,き裂のある供試体ではき裂面に向けて,き裂の無い供試体ではルートギャップの角部に向けて集束する散乱波が明瞭に現れている.これらの散乱波成分の位置を,時間を追って調べると,概ねFig.4に白の破線で示した経路をたどる.(a)のケースでは,フランジ底面で2回の反射後,余盛り表面を経由した波がき裂面に集束しており(b)と経路が大きく異なる.このことは,観測波形を適切に処理すれば,反響環境下でもき裂検出と位置推定が可能であることを示唆する.ただし,従来のイメージング法では散乱波の経路を予め把握しておく必要があり,多数の経路が想定される今回のような条件では適用が困難である.一方,時間反転法では,散乱源に集束する波動場を得るにあたり,伝播経路や伝播モードに関する仮定や事前情報を必要とせず,その点に大きなメリットがある.Fig.4はこのことを例示する良好な結果と言える.
