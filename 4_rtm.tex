最後に,reverse-time migration(RTM)の考え方に基づき,きず(スリット)の
超音波イメージングを行った結果を示す.
\subsection{イメージング方法}
RTMでは入射場と時間反転場の相関を
取る形式の画像化関数を定義し,弾性係数や密度のコントラストを再構成する.
例えば,イメージングに速度場を用いる場合,最も単純な画像関数$I(\fat{x})$は,
入射変位場を$u_i^{in}(\fat{x},t)$として次式で与えられる.
\begin{equation}
	I(\fat{x})=\int_T \dot u^{in}_i(\fat{x},t) \dot u_i^{\dagger}(\fat{x},t)dt
	\label{eqn:Ix}
\end{equation}
これは,FWIにおけるFr$\acute{\rm e}$chet kernelの一つ,
式(\ref{eqn:K_rho})の$K_\rho(\fat{x})$において,
$u_i=u_i^{in}$としたものに他ならない.
つまり,この場合のRTMは,FWIにおける目的関数の勾配を画像化関数として
用ていることになる.ただし,RTMでは他にも様々な画像化関数が提案されているため\cite{Jones},
RTM一般が,FWIの特別なケースということではない.
本節では,$K_\rho(\fat{x})$に加え,FWIのもう一つのカーネル
$K_{ijkl}(\fat{x})$を用いてイメージングを行う.
$K_{ijkl}(\fat{x})$は4階のテンソルだが,
等方性体の場合,式(\ref{eqn:Cijkl_iso})を用いれば,
\begin{equation}
	\int_G K_{ijkl}C_{ijkl}dv
	=
	\int_G \lambda u_{i,i}u^{\dagger}_{j,j} dv
	+
	\int_G \mu u_{i,j}u^{\dagger}_{i,j} dv
	\label{eqn:}
\end{equation}
となる.つまり,$K_{ijkl}$の独立な成分はラメ定数に対応した二つである.
それらをあらためて
\begin{equation}
	K_\lambda(\fat{x}):= \int_T u_{i,i}u^{\dagger}_{j,j} dt
	\label{eqn:K_lmb}
\end{equation}
\begin{equation}
	K_\mu(\fat{x}):= \int_T u_{i,j}u^{\dagger}_{i,j} dt
	\label{eqn:K_mu}
\end{equation}
と定義する.以下に示すRTMイメージングでは,これら
$K_\rho, K_\lambda$および$K_\mu$において,$u_i(\fat{x},t)=u_i^{in}(\fat{x},t)$
としたものも画像化関数として用いる.
\subsection{入射場の計算}
RTMイメージングに用いた入射速度場$\dot{u}_i^{in}(\fat{x},t)$の挙動を
{\bf 図}-\ref{fig:inc_btm}に示す.これは,{\bf 図}-\ref{fig:spec}の
試験体と同じ厚さ(12mm)のアルミニウム平板を用い,別途計測した速度波形を利用して
FDTD計算で求めたものである.
平板を使った実験は,超音波探触子からの入射場を把握することを目的としたもので,
{\bf 図}-\ref{fig:experiment2}のように,探触子設置面と反対の側からLDV
で表面振動を計測した.LDV計測は{\bf 図}-\ref{fig:experiment2}に示す
幅65mmの範囲$S'$において0.2mmピッチで行い,得られた速度波形を使い
FDTD計算で入射場を再構成する.{\bf 図}-\ref{fig:fd_model}の$S'$は,
その際,速度波形を与えて加振したFDTDモデル上の位置を表す.
つまり,ここでは,実際の送信源である超音波探触子の
設置面$S$でなく,$S'$に相当する位置で実測した波形を擬似的な
振源項として与え,FDTD法で試験体内部の入射場を求めている.
そのため,探触子から入射された直後に左下方向へ進む波動場はFDTD計算上
再現されず,{\bf 図}-\ref{fig:inc_btm}では,底面で一回反射した
後の入射場の様子が現れている.
\begin{figure}[thb]
\centering
	\includegraphics[clip,scale=0.4]{Figs/IncBtm.eps}
	\caption{計測波形からFDTD法で再構成した入射速度場}
	\label{fig:inc_btm}
\end{figure}
\begin{figure}[hbt]
\centering
	\includegraphics[clip,scale=0.35]{Figs/experiment2.eps}
	\caption{平板試験体を使った速度波形の計測}
	\label{fig:experiment2}
\end{figure}
このように,計算上は再現されない一部の入射波成分が存在するが,
実際の探触子位置($S$)は斜角入射された横波がプレート下面で一回反射した後,
切り欠き角部に到達するよう設定している.従って,現実の送信源である$S$から
スリットに向けて,プレート下面を経由せずに到達する実体波は弱く,散乱波の発生にほとんど寄与しない.
このことから,スリットへの直達波が再構成された入射場に含まれないことは,
今回のRTMイメージングでは問題とならない.
%%%%%%%%%%%%%%%%%%%%%%%%%%%%%%%%%%%%%%%%%
\subsection{イメージング結果}
{\bf 図}-\ref{fig:inc_btm}の入射場と{\bf 図}-\ref{fig:snap_crack}の
時間反転場から求めた$K_\rho(\fat{x}),K_{\lambda}(\fat{x})$および
$K_{\mu}(\fat{x})$を{図-}\ref{fig:imgs}に示す. 
この図には,3種類のRTMイメージがそれぞれの最大値で無次元化して表示されている.
ただし,散乱源の存在は画像化関数が正の値を取る箇所として示されるため,
画素値の表示範囲は0から1としている.
\begin{figure}[tbh]
\centering
	\includegraphics[clip,scale=0.6]{Figs/imgs.eps}
	\caption{計測波形を用いて行った3種類のRTMイメージングの結果}
	\label{fig:imgs}
\end{figure}
白の実線はFDTDモデル,すなわち,試験体の外形を示し,
紫の実線は後に示す{\bf 図}-\ref{fig:imgs_zoom}において拡大表示する
範囲を表している.なお,RTMイメージングは{\bf 図}-\ref{fig:fd_model})
の解析領域全体で行っているが,{図-}\ref{fig:imgs}に示す継手周辺以外の場所では,
画像化関数の値が小さく,有意な指示が現れないことを確認している.
\\
\hspace{\parindent}
これら3種類のRTMイメージのうち$K_\rho(\fat{x})$と$K_{\mu}(\fat{x})$では,
切り欠き側面とスリット開口部付近に大きな画素値をもつ箇所がある.
{\bf 図}-\ref{fig:imgs_zoom}は,その状況をより詳しく観察するために
{\bf 図}-\ref{fig:imgs}で紫の実線で囲った箇所を拡大したものである.
これら拡大図ではイメージング結果とスリット位置を対照するため,
スリット位置を白の破線で描き加えてある.
{\bf 図}-\ref{fig:imgs_zoom}の(a)と(c)に示される通り,
$K_\rho$と$K_\mu$では,大きな画素値をもつ箇所が,実際の散乱源である
スリットと切り欠き側面に非常によく一致している.
\begin{figure}[tbh]
\centering
	\includegraphics[clip,scale=0.6]{Figs/imgs_zoom.eps}
	\caption{計測波形を用いて行った3種類のRTMイメージングの結果
	(スリット周辺を拡大して表示)}
	\label{fig:imgs_zoom}
\end{figure}
一方,スリットの下側は画像に現れず,これはスリット下端側からの
散乱波が元々観測波形に含まれていないことが原因と考えられる.
実際,{\bf 図}-\ref{fig:inc_btm}において入射場の挙動を見ると,
プレート底面側から来る横波はスリット下端部を逸れ,{\bf 図}-\ref{fig:inc_btm}(f)では
切り欠きの上側で大きな振幅を持つことが確認できる.
このことを考慮すれば,観測波形に含まれる横波散乱波は微弱なものであるにも
関わらず,時間反転場の計算を通じて効果的にイメージングに利用されていると言える.
ただし,継ぎ手周辺には,相対画素値が0.2から0.4程度となる箇所がスリット
近傍以外にも多数見られ背景のノイズとなっている.これは,継ぎ手内部の残響に
よるため,完全に抑制することは難しい.しかしながら,送受信位置と観測時間範囲には
最適化の余地が大きく,実験上も圧電センサーでの受信を行えば信号強度は大きく
改善できる.そのため,RTMイメージの信号−雑音比を向上させることは今後十分に期待がもてる.
\\
\hspace{\parindent}
$K_\rho(\fat{x}), K_\mu(\fat{x})$が良好な結果を与えるのに対し,
$K_\lambda(\fat{x})$は,スリットの有無を判断できるほどの結果が
現れていない.これは,今回検出された散乱波がスリットで生じた横波散乱波
であることが理由と考えられる.なぜなら,$K_{\lambda}(\fat{x})$はFWIの
観点ではラメ定数$\lambda$に対する目的関数の感度を意味し,
$\lambda$は縦波速度に関係する弾性係数だからである.従って,
スリットで発生した縦波散乱波が観測波形に含まれない限り,$K_\lambda(\fat{x})$
が高い感度を示さないことは必然的な結果と言える.
%\\\hspace{\parindent}
%これまでに示した$K_\rho(\fat{x})$や$K_\mu(\fat{x})$のような超音波イメージを,波形の遅延−重ね合わせに基づく方法で
%得ることは以下の理由から困難と考えられる.第一に,遅延時間の設定には散乱波の経路特定が必要だが,
%時間反転場の計算を行うことなく,今回のように複数回の反射とモード変換を伴うエコー経路を特定することは難しい.
%第二に,経路が正しく特定されとしても,対応するエコー波形をそのまま画像化領域に投影する方法では,エコー
%継続時間がそのまま像の滲みになって現れる. 今回の計算で言えば,{\bf 図}-\ref{fig:snap_crack}(f)
%において赤色で示された波群が占める領域程度の滲みは避けられない.第三に,画像解像度の向上にデコンボリューションなどの
%パルス圧縮技術を援用することも簡単ではない.なぜなら,デコンボリューションには対象波形と類似したスペクトル成分
%を持つ参照波形が必要とされるが,そのような参照波形を経路枚に用意することは現実的ではない.以上より,時間反転法は,
%複雑形状部位の超音波イメージングにおいて,従来法を超える適用範囲と性能が期待できるという点で有望なアプローチと考えられる.

