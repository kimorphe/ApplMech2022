%\section{RTMイメージング}
\subsection{イメージング方法}
最後に,reverse time migration(RTM)の考え方に基づき,きずの超音波イメージング
を行った結果を示す.RTMでは入射場と時間反転場の量の相関を取る形式の
画像化関数を定義し,媒体の弾性係数や密度のコントラストを可視化する.
例えば,イメージングに速度場を用いる場合,最も単純な画像関数$I(\fat{x})$は
入射変位場を$u^{in}(\fat{x},t)$として次式で与えられる.
\begin{equation}
	I(\fat{x})=\int_T \dot u^{in}_i(\fat{x},t) \dot u_i^{\dagger}(\fat{x},t)dt
	\label{eqn:Ix}
\end{equation}
これは,FWIにおけるFr$\acute{\rm e}$chet kernelの一つである式(\ref{eqn:Frechet_rho})
の$K_\rho(\fat{x})$において$u_i=u_i^{in}$とした場合に他ならない.
つまり,この場合RTMではFWIにおける勾配を画像化関数,すなわち欠陥検出指標として用い
ると言える.ただし,RTMでは他にも様々な画像化関数が提案されているため,
一般にはRTMはFWIの特別な場合ということはできない.
本節では,FWIのもう一つのカーネルである$K_{ijkl}(\fat{x})$も用いて,
イメージングを行った結果を示す.$K_{ijkl}(\fat{x})$は4階のテンソルだが,
等方性体の弾性係数(\ref{eqn:Cijkl_iso})を用いれば,
\begin{equation}
	\int_G K_{ijkl}C_{ijkl}dv
	=
	\int_G \lambda u_{i,i}u^{\dagger}_{j,j} dv
	+
	\int_G \mu u_{i,j}u^{\dagger}_{i,j} dv
	\label{eqn:}
\end{equation}
と表すことができる.つまり,$K_{ijkl}$の独立な成分はそれぞれの
ラメ定数に対応する二つである.それらをあらためて
\begin{equation}
	K_\lambda(\fat{x}):= \int_T u_{i,i}u^{\dagger}_{j,j} dt
	\label{eqn:K_lmb}
\end{equation}
\begin{equation}
	K_\mu(\fat{x}):= \int_T u_{i,j}u^{\dagger}_{i,j} dt
	\label{eqn:K_mu}
\end{equation}
と定義し,$u_i(\fat{x},t)$には入射場$u_i^{in}(\fat{x},t)$を与えて
イメージングを行う.以下,式(\ref{eqn:K_rho}),(\ref{eqn:K_lmb}),
(\ref{eqn:K_mu})に入射場$u_i^{in}(\fat{x},t)$を用いて行った
イメージング結果について述べる.
\subsection{入射場の計算}
上記,RTMイメージングに用いた入射速度場の進展挙動を{\rm 図}-\ref{fig:inc_btm}
に示す.この結果は,{\rm 図}-\ref{fig:sample}に示した試験体と同じ厚さ(12mm)の
アルミニウム平板を用いた計測結果から計算したものである.平板を使った実験では,
超音波探触子からの入射波を平板裏面(探触子を設置した面と逆側の面)側からLDVで
計測した波形を用いてFDTD法で計算したものである.
平板試験体での計測は,入射波の進展方向へ65mmの範囲を0.2mmピッチで計測し,
その結果得られた速度波形を,{\rm 図}-\ref{fig:fd_model}で$S'$として
示した範囲を加振する表面速度波形として与えた.つまり,実際の送信源である
超音波探触子の設置面$S$でなく,$S'$に相当する位置で別途平板で観測した
波形を擬似的な振源項として与え試験体内部の入射場を求めた.
そのため,{\rm 図}-\ref{fig:inc_btm}では,探触子から入射された直後の
左下方向への波動場は再現されず,それらが一回底面で反射した後の挙動が現れている. 
しかしながら,探触子からの入射角は概ね65から70度程度で,探触子位置($S$)は,
プレート下面で一回反射した横波が切り欠きのコーナー部に到達するよう設定している.
そのため,物理的な送信源$S$からスリットへプレート表面を経由せずに直接到達する
実体波は非常に弱く、散乱波の発生にもほとんど寄与しないと考えらる.
以上のことから,探触子からスリットへの直達波が計算上の入射場に
含まれないことは,次に示すRTMイメージング上で問題とならない.
\begin{figure}[htb]
\centering
	\includegraphics[clip,scale=0.4]{Figs/IncBtm.eps}
	\caption{計測波形から再構成した入射場(スリットがない場合)}
	\label{fig:inc_btm}
\end{figure}
%%%%%%%%%%%%%%%%%%%%%%%%%%%%%%%%%%%%%%%%%
\subsection{イメージング結果}
{\rm 図}-\ref{fig:inc_btm}に示した入射場と,{\rm 図}-\ref{fig:snap_crack}に
示した時間反転場の解析結果を用いて計算した
$K_\rho(\fat{x}),K_{\lambda}(\fat{x})$および$K_{\mu}(\fat{x})$を
{図-}\ref{fig:imgs}に示す. これらは,3種類のRTM画像をそれぞれの最大値で
無次元化した値を示したもので,白の実線はFDTDモデル,すなわち試験体の外形
を示している.FDTD法による波動場の計算では,スリットの無いモデルで計算を
行っているが,画像化結果には白の破線でスリット位置も記入してある.
なお,RTMイメージングは,FDTDモデル全体で行っているが,{図-}\ref{fig:imgs}に
示した継手部周辺の外側ではほとんど画素値をもたず有意なし指示は現れない.
3種類のRTMイメージのうち,$K_\lambda(\fat{x})$では,スリットおよび切り欠きの
周辺で比較的大きな値をもつものの,スリットが存在すると判断できるほど
明確なものではない.
一方,$K_\rho(\fat{x})$と$K_{\mu}(\fat{x})$は,切り欠きの側面
と,スリットの概ね上半分の位置に明らかな指示が現れ,スリットの存在する
ことを示している.
この結果は,スリットからの散乱波がS波であるため、
体積弾性係数への寄与が大きいラメ定数$\lambda$に対する感度が低いことを示唆している.
大きな画素値をもつ位置は,実際の散乱源である
スリット表面、切り欠き側面と非常に正確に一致していることも分かる.
もともとのエコー強度が弱く,継手内部で反響してとどまる成分が
多いことから、画像の背景ノイズも大きいが,赤や黄色で示された
相対的に大きな画素値は正確に検出すべきスリットと切り欠き側面を示している.
このような結果を開口合成法で得ることは非常に困難で、
時間反転法を用いることに十分な意義があることを示している.
開口合成法では、第一に経路の特定が必要だが、時間反転場の計算を行うことなく
このような複数回反射しモード変換を伴うエコー経路を特定することはできない。
また、経路が正しく特定され、対応するエコー波形を抽出することができたとして、
エコー波形をそのまま画像化領域に投影するイメージング方法では、
エコー継続時間がそのままきず指示のにじみとなって現れる。
今回の計算結果で言えば、{\rm 図}-\ref{fig:snap_crack}(f)において
赤で示された波群程度のにじみは避けられず,RTMイメージのような
像合成は期待できない.開口合成像の解像度をあげるために,
参照波形を使ったデコンボリューシュ処理によりパルス圧縮で
前処理した波形を像合成に用いることは考える.しかしながら、
このようなアプローチでも,溶接ビードを経由するようなエコー波形は
底面エコーのような波形とは大きく異なるため,参照波形自体を
用意することがそもそも非常に困難である.
これらのことを踏まえると,{図-}\ref{fig:imgs}に示したRTMイメージには
改善の余地は大きいものの、同様な解像度や正確さをもつ画像を
従来のSAFTやその類似法で実現することは極め困難であると言える.
\begin{figure}[htb]
\centering
	\includegraphics[clip,scale=0.6]{Figs/imgs.eps}
	\caption{計測波形から合成したRTM画像}
	\label{fig:imgs}
\end{figure}
