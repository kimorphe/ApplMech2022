超音波探傷法は,部材内部や表面のき裂や空洞をはじめとするきずの検出に用いられる
非破壊検査法の一つである.超音波探傷試験では,計測結果としてエコー強度の
時刻歴波形が得られるため,き裂位置や大きさを推定する際にはしばしば波形データ
を用いた画像合成が行われる.その代表的な方法にはフェーズアレイ法や開口合成法,
total focusing method, 線形化逆散乱法等が挙げられる.これらの方法はいずれも,
超音波の伝播経路を仮定してエコー波形を画像化領域に投影するものである.
従って散乱波の伝播経路が特定できることを前提としたイメージング手法ということができる.
線形化逆散乱法では経路特定のプロセスが明示的には現れないが, 全空間のグリーン関数,
より正確にはその遠方近似解を用いてた定式化が行われるているため,
実質的には経路が特定できることを前提とした方法になっている.
\\
\hspace{\parindent}
現実の構造物における超音波試験では,固体内部での超音波の反射や回折,
モード変換のために,伝播経路の特定は必ずしも容易でない.
特に,板材や部材継手周辺では,入射波や散乱波の反響や回折のために想定しうる
伝播経路が多数存在し,きずからのエコーを分離して観測することや,経路を事前あるいは
事後的に特定することが難しい.例えば,U型閉断面リブを有する鋼床版では,
Uリブとデッキプレートの隅肉溶接部で発生する疲労き裂による損傷が問題となっている.
そのため,超音波探傷によるき裂検出が望まれているが,Uリブとデッキプレートは
いずれも10mm程度の板厚しかなく,入射波,散乱波とも伝播途上で繰返し反射される.
また,溶接部の形状に起因したエコーの発生源が複数存在すること,
継手に対してセンサーが設置可能な範囲が制限されること
などの理由から,元来微弱なき裂エコーの検出や経路特定が難しい問題になっている.
このような状況では,開口合成法に代表される伝播経路の特定を前提とした従来の
超音波イメージングの方法は適用しにくい.
\\
\hspace{\parindent}
これに対して時間反転集束の原理に基づく各種のイメージング手法は,一般に
計算負荷は高いものの入射波や散乱波の伝播経路を予め特定する必要がなく,
上述のような溶接継手部の超音波探傷に有用となることが期待できる.
時間反転集束法では,観測波形を時間反転して媒体に再入力したとき,
散乱波がその発生源に向かって集束する性質を利用する.
この性質は材料減衰が無視できる場合,波動場の支配方程式が時間反転に対して
不変であることに由来するもので,未知のターゲットへ波動場を集束させることや,
時間反転場を計算で可視化することによってターゲットを発見するといった
利用の仕方がある.このようなアイデアは古くから知られており,
Finkらの先駆的な研究によって実験的にも理論的にもその有効性が示されている.
他にも,時間反転場はreverse time migration (RTM), フルウェーブ・インバージョン
トポロジー勾配(導関数)法といった波動問題のインバージョンにおいて広く現れる.
例えば,物理探査分野でしばしば用いられるRTMでは,入射場と時間反転場の
相関をとることで速度構造の推定が行われる. これは、時間反転場と入射場の位相が,
反射源となる地層境界(速度不連続面)で一致することを利用した方法と理解することが
できる.一方,フルウェーブ・インバージョンやトポロジー勾配法では,観測波形と
シミュレーション波形の差を最小化するように,媒体の物性や形状を推定する.
これらは随伴方程式法による最適化法の一種で,目的関数の推定式には随伴問題の解(随伴場)
が含まれる.随伴場は物理的には時間反転場に相当するため,
ここでも時間反転場が散乱源に集束する性質が利用されているといえる.
\\
\hspace{\parindent}
これら時間反転集束の性質に関係するインバージョン法は,地震学や物理探査,
医療超音波分野への応用では実データに対して用いられその有効性が実証されている.
一方,超音波探傷への適用も検討されているが,計測波形を使った検討は
これまでほとんど用いられていない。 
超音波探傷試験では,地盤や生体と異なり,無限あるいは半無限領域とみなせる
状況は稀で,媒体境界の影響があることが避けられない.
そのため,探査で使われている手法がそのまま超音波探傷にも有効であるかどうかは
明らかでなく,特に溶接継手部のような複雑に反響する波動場が関与する問題で、
RTMを始めとする時間反転集束法が十分機能するかは,実測データも利用して
検証を行う必要がある.
%
以上のことを踏まえ本研究では、時間反転集束に基づくRTMや随伴方程式法が,
超音波探傷試験におけるイメージングにも有効なものとなり得るかを検討することを
目的とし,実験で得た超音波波形を使い時間反転場の集束挙動について調べる.
実験にはUリブの隅肉溶接部を想定した試験体を使い,模擬き裂からのエコーを
レーザー振動計で多点計測する.観測した波形は時間反転して差分法による
波動解析モデルに入力し,時間反転場の進展挙動を調べる.これらの実験と
計算結果を元に,超音波探傷に利用可能なエコー成分が検出できること,
RTMイメージングによって正しい位置にきずの指示を結像させることが可能で
あることを示し,時間反転法が反響環境下での超音波探傷にも有効であることを実証する.
