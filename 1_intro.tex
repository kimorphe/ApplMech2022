花崗岩はき裂性の結晶質岩で,新鮮な状態でも造岩鉱物粒程度のスケールでは多数のマイクロクラックを有することが知られている\cite{Kudo1}$^-$\cite{Takagi}.
マイクロクラックは花崗岩の透水性や物質輸送特性,剛性,強度,物理・化学的な風化の進行に影響を与える.
そのため,マイクロクラックの大きさや量を把握することは,例えば,高レベル放射性廃棄物の地層処分のように,
長期に渡って安全性を担保する必要のある処分場を花崗岩体中に建設する際に重要となる.
%
岩石中のき裂の大きさや量,分布状況は偏光顕微鏡で岩石薄片を観察することで調べられる.
しかしながら,岩石薄片の作成には手間がかかる上,岩石内部の状態やき裂の経時的な変化を観察することはできない.
一方,岩石の弾性波試験は結果の定量的解釈が難しいことが多い点に難があるものの,試料内部の状態を簡単な検査で調べることができ,
経時的なモニタリングにも適用できるという面で利点がある.
実際,弾性波の伝播はき裂の影響を強く受けるため,弾性波計測結果からき裂の量や進展に関する情報を得る
試みが行われている\cite{Griffiths}.
ただし,花崗岩を始めとする岩石中の弾性波伝播は鉱物粒やき裂による多重散乱のために非常に複雑で\cite{Sato},
弾性波計測データから鉱物粒径やき裂を定量的に評価する方法は確立されていない.
%
花崗岩中の弾性波伝播特性を理解するためには,多数の鉱物粒から成る多結晶質体モデルを用いて
弾性波の多重散乱解析を行うことが有効となる\cite{FEM}.
そのような多重散乱解析の実施には造岩鉱物の密度と弾性係数値を与える必要がある.
これらの物性値のうち密度は,物質の結晶構造と構成原子の質量から鉱物種ごとに
定めることができ,鉱物種間での差も小さい.
各種造岩鉱物の弾性係数も鉱物学分野で古くから調べられており,文献値\cite{AGU}や
分子シミュレーション\cite{Kawamura}の結果から知ることができる.ただし,それらはき裂の無い
単結晶鉱物に対するものであるため,岩石を構成する鉱物粒の弾性係数としてそのまま用いることはできない.
また,弾性係数は結晶軸方向やき裂の有無による不確実性があるため,対象とする岩石の状態を反映して
確率的に与える必要もある.

花崗岩の弾性波速度や弾性係数に関する研究は,これまでに多数行われている.
例えば,工藤ら\cite{Kudo2},佐野ら\cite{Sano1}は花崗岩コア試料の超音波透過試験を行い,
花崗岩が直交異方性体として振る舞うことを示し,マイクロクラックとの関係を検討している.
ただし,それらの研究では岩石コアスケールでのマクロな異方性や弾性波速度が論じられており,
鉱物粒スケールでの弾性係数や弾性波速度を調べたものではない.
これに対してSivaji\cite{Sivaji}およびFukushima\cite{Fukushima}らは,
レーザードップラー振動計を使って花崗岩試料表面に励起された
超音波振動を可視化し,鉱物粒径と弾性波到達時間のゆらぎの関係を調べている.
岩石のランダムな不均質性に起因した弾性波到達時間のゆらぎについてはRythov近似に基づく方法で
理論的な検討が加えられ,到達時間のゆらぎと不均質性の空間スケールの関係が得られることが示されている
\cite{Muller}$^-$\cite{Baig}.
ただし,これらの研究において、弾性波速度の鉱物粒スケールでの変動は,ガウス分布や指数分布
等、取扱が容易なよく知られた確率分布に従うことが仮定されており,造岩鉱物粒の弾性波速度が従う
実際の確率分布はこれまで調べられていない.

そこで本研究では,超音波計測によって局所的な縦波(P波)の位相速度を統計的に求めることを試みる.
これを多数の鉱物粒について鉱物種毎に行うことで,P波位相速度が従う確率分布を評価する.
以下では,この目的のために行った超音波計測方法について述べた後,実験で得られたP波位相速度分布
の結果を報告する.具体的には,1$\sim$2MHzの周波数帯域におけるP波位相速度の頻度分布を示し,
結晶粒のスケールでは鉱物種に依らず速度分散はほとどんど無いこと,
長石と石英ではP波位相速度の分布が有意に異なることを示す.
また,これら実験で得られたP波位相速度を用いた数値波動解析を行い,縦波平面波の散乱減衰挙動
について検討を行う.その結果,P波位相速度の頻度分布を単一の正規分布と仮定した場合と,
計測から得た頻度分布を用いた場合で散乱減衰の挙動が異なることを示す.
以上の結果から,弾性波媒体として花崗岩をモデル化する上で,実際に計測を行って評価した
位相速度の分布を反映したモデルを用いることが必要であることを述べる.
なお,以下ではP波位相速度を単に位相速度と呼ぶ.

