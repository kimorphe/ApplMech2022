超音波探傷法は,部材内部や表面き裂の検出に用いられる非破壊検査法の一つである.超音波探傷試験では,エコー強度の時刻歴波形が計測結果として与えられるため,き裂位置や大きさを推定する際,しばしば,画像合成が行われる.代表的な方法にはフェーズアレイ法や開口合成法が挙げられる.これらの方法は,いずれも超音波の伝播経路を仮定してエコー波形を画像化領域に投影する.しかしながら,固体内部では反射や回折,モード変換が生じ超音波伝播経路の特定は必ずしも容易ではない.特に,板材や部材継手部周辺では,入射波や散乱波の反響により想定しうる伝播経路が多数存在し,きずからのエコーを分離して観測することや,その経路特定は難しく,従来の方法ではきず画像の合成も困難になる.これに対して,時間反転集束の原理に基づく画像化法1)では,計算負荷は高いものの入射波や散乱波の伝播経路を特定する必要がないという利点がある.時間反転集束実験はこれまで行われているが,固体の超音波探傷を想定した検討はほとんど行われていない.特に,多重反射の効果が顕著で散乱波の伝播経路特定が困難となる反響環境下で,き裂からの微弱なエコーを散乱源に正しく時間反転集束させることができるか否かは明らかでない.本研究では,この点を明らかにすることを目的として,超音波計測と数値解析によって散乱波の時間反転集束挙動を調べる.以下では,模擬き裂からの超音波エコー計測に関する実験概要を示し,次に,時間反転場の計算方法を述べる.その後,時間反転場の計算結果を示し,き裂位置に集束する散乱波が良好に再構成できることを示す. 
