超音波探傷法は,部材内部や表面の欠陥検出に用いられる非破壊検査法の一つである.
超音波探傷試験では計測結果としてエコー強度の時刻歴波形が得られ,
きず位置や大きさを推定する際には,しばしば波形データから画像合成(イメージング)
が行われる.代表的なイメージング法にはフェーズアレイ法\cite{Wilcox2007}や
開口合成法\cite{Thomson1984, Doctor1986,Langenberg1986, Schmitz2000,Spies2012},
total focusing method\cite{Holmes2008, Zhang2010}, 
線形化逆散乱法\cite{Langenberg1989, Kitahara2002, Shlivinski2007}
といった方法がある.これらはいずれも超音波伝播経路を仮定し,エコー波形を画像化領域
に投影するトモグラフィー的手法で,散乱波の伝播経路を特定できることが適用の前提となる.
%
しかしながら,実際の超音波探傷試験では,部材内での超音波の反射や回折,
モード変換のために,伝播経路の特定が難しいことも少なくない.特に,部材の
継手周辺では,入射波や散乱波の回折や反響のために想定しうる伝播経路が
複数存在し,きずエコーを分離して観測することや,主要経路を特定することが難しい.
例えば,U型閉断面リブを有する鋼床版では,隅肉溶接部で発生する疲労き裂
による損傷が問題となっており,超音波探傷によるき裂検出が望まれている
%\cite{Urib1,Urib2, Urib3}.
\cite{Urib3}.
しかしながら,リブやデッキプレートの板厚は10mm程度で,超音波は伝播途上で
繰返し反射される.加えて,継手形状に起因したエコー発生源も複数存在し,
探傷位置に関する制約も強い.そのため,元来微弱なき裂エコーの検出や
経路特定が難しく,トモグラフィー的な超音波イメージング法は適用しにくい.
\\
\hspace{\parindent}
これに対して時間反転集束の原理に基づく各種のイメージング法\cite{FinkTextBook}は,
入射波や散乱波の伝播経路を予め特定する必要がなく,上述のような溶接継手
の超音波探傷にも有効であることが期待できる.
時間反転集束法では,観測波形を時間反転して媒体に再入力したとき,散乱波が
その発生源に向かって集束する性質を利用する.この性質は波動方程式が
時間反転不変であることに由来するもので,未知のターゲットへ波動場を集束させる
ことや,時間反転場を計算で可視化することによってターゲットを発見するといった
利用の仕方がある.このようなアイデアは古くから知られており,
Finkらの先駆的な研究\cite{Fink1992, Prada1994, Prada1995, Prada1995_2,  Mordant1999}
によって実験的にも理論的にもその有効性が示されている.時間反転場は他にも,
reverse-time migration(RTM)\cite{Yan2008, Etgen2009, Velichko2010, Chung2012, Jones,KK_RTM}, 
フルウェーブインバージョン\cite{Fichtner, Talantola1984},
トポロジー勾配(導関数)法\cite{Dominguez2005, Dominguez2010, Gibiat2010, Bonnet2008, Saitoh2021}
といった波動問題の逆解析において広く現れる.例えば,物理探査分野でしばしば用いられるRTMでは,
入射場と時間反転場の相関をとることで地下の速度構造推定が行われる. 
また,フルウェーブ・インバージョンやトポロジー勾配法では,観測波形とシミュレーション
波形の差を最小化するように媒体の物性や形状を推定する.
これらは随伴方程式法による最適化法の一種で,目的関数の最小化に
随伴問題の解(随伴場)を利用する.随伴場は物理的には時間反転場に相当するため,
これらの方法でも時間反転場が散乱源に集束する性質が利用されていると言える.
\\
\hspace{\parindent}
%%%%
%%% Medical Imaging の文献_
これら時間反転集束の性質を利用したインバージョン法は,物理探査\cite{Fichtner, Etgen2009}や
医療超音波分野\cite{FinkTextBook, Tanter2000}への応用では実データに対して適用され,
その有効性が実証されている.これに対して超音波探傷分野では,
シミュレーション波形を使ったイメージング例\cite{KK_RTM,Saitoh2021}はあるものの,
計測波形を使った検討や実際の探傷への利用\cite{Nakahata2019}はあまり進んでいない.
地盤や生体と異なり,超音波探傷試験では媒体を無限あるいは半無限領域とみなせる
状況は稀で,イメージングにおいて境界の影響は無視できない.
そのため,開領域を対象とすることの多い探査等の分野で使われている手法が,
そのまま超音波探傷にも有効である
かどうかは明らかでない.とりわけ,溶接継手部のような波動場の反響が想定される
問題に対し,RTMを始め時間反転場の集束に成否が依存するイメージング法が
十分機能するかは,実測データも利用した検証が必要である.
\\
\hspace{\parindent}
以上を踏まえ本研究では,時間反転場の集束に基礎を持つイメージング法が
超音波探傷試験に対しても有用であるかを知るべく,反響環境下での時間反転場の
集束挙動を実験と数値シミュレーションで調べる.実験にはUリブの隅肉溶接部を想定した
試験体を使い,模擬き裂からのエコーをレーザー振動計で計測する.計測した波形は時間
反転して差分法波動解析モデルに入力し,時間反転場の進展挙動を調べる.
これらの実験と計算を通じ,時間反転場の計算によってはじめて見出される有用な
エコーが存在すること,またそれを用いたRTMイメージングで正しい位置に指示をもつ欠陥
画像が合成できることを示す.以上により,時間反転法が超音波探傷試験においても有用であることを示す.
%線形化逆散乱法では経路特定のプロセスが明示的には現れないが, 
%全空間のグリーン関数,より正確にはその遠方近似解を用いてた定式化が行われるているため,
%実質的には経路が特定できることを前提とした方法になっている.

