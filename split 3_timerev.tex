%\section{RTMイメージング}
\subsection{イメージング方法}
最後に,reverse time migration(RTM)の考え方に基づき,きず(スリット)の
超音波イメージングを行った結果を示す.RTMでは入射場と時間反転場に関する
量の相関を取る形式で画像化関数を定義し,媒体の弾性係数や密度のコントラスト
を再構成する.例えば,イメージングに速度場を用いる場合,
最も単純な画像関数$I(\fat{x})$は,入射変位場$u^{in}(\fat{x},t)$として
次式で与えられる.
\begin{equation}
	I(\fat{x})=\int_T \dot u^{in}_i(\fat{x},t) \dot u_i^{\dagger}(\fat{x},t)dt
	\label{eqn:Ix}
\end{equation}
これは,FWIにおけるFr$\acute{\rm e}$chet kernelの一つ,式(\ref{eqn:Frechet_rho})
の$K_\rho(\fat{x})$において$u_i=u_i^{in}$としたものに他ならない.
つまり,この場合RTMではFWIにおける目的関数の勾配を,画像化関数すなわち
欠陥検出指標として用いることになる.ただし,RTMでは他にも様々な画像化関数が
提案されているため,一般に,RTMがFWIの特別なケースになるということではない.
本節では,FWIのもう一つのカーネルである$K_{ijkl}(\fat{x})$を用いて
行ったイメージング結果も示す.$K_{ijkl}(\fat{x})$は4階のテンソルだが,
等方性体の場合,式(\ref{eqn:Cijkl_iso})を用いれば,
\begin{equation}
	\int_G K_{ijkl}C_{ijkl}dv
	=
	\int_G \lambda u_{i,i}u^{\dagger}_{j,j} dv
	+
	\int_G \mu u_{i,j}u^{\dagger}_{i,j} dv
	\label{eqn:}
\end{equation}
となる.つまり,$K_{ijkl}$の独立な成分は
ラメ定数に対応して二つである.それらをあらためて
\begin{equation}
	K_\lambda(\fat{x}):= \int_T u_{i,i}u^{\dagger}_{j,j} dt
	\label{eqn:K_lmb}
\end{equation}
\begin{equation}
	K_\mu(\fat{x}):= \int_T u_{i,j}u^{\dagger}_{i,j} dt
	\label{eqn:K_mu}
\end{equation}
と定義する.RTMイメージングには,これらのカーネルにおいて$u_i(\fat{x},t)=u_i^{in}(\fat{x},t)$
としたものを,画像化関数に用いる.以下,式(\ref{eqn:K_rho}),(\ref{eqn:K_lmb}),
(\ref{eqn:K_mu})に入射場$u_i^{in}(\fat{x},t)$を用いて行った
イメージング結果について述べる.
\subsection{入射場の計算}
上記,RTMイメージングに用いた入射速度場の進展挙動を{\rm 図}-\ref{fig:inc_btm}
に示す.この結果は,{\rm 図}-\ref{fig:sample}に示した試験体と同じ厚さ(12mm)の
アルミニウム平板を用いた計測結果から計算したものである.平板を使った実験では,
超音波探触子からの入射波を平板裏面(探触子を設置した面と逆側の面)側からLDVで
計測した波形を用いてFDTD法で計算したものである.
平板試験体での計測は,入射波の進展方向へ65mmの範囲を0.2mmピッチで計測し,
その結果得られた速度波形を,{\rm 図}-\ref{fig:fd_model}で$S'$として
示した範囲を加振する表面速度波形として与えた.つまり,実際の送信源である
超音波探触子の設置面$S$でなく,$S'$に相当する位置で別途平板で観測した
波形を擬似的な振源項として与え試験体内部の入射場を求めた.
そのため,{\rm 図}-\ref{fig:inc_btm}では,探触子から入射された直後の
左下方向への波動場は再現されず,それらが一回底面で反射した後の挙動が現れている. 
しかしながら,探触子からの入射角は概ね65から70度程度で,探触子位置($S$)は,
プレート下面で一回反射した横波が切り欠きのコーナー部に到達するよう設定している.
そのため,物理的な送信源$S$からスリットへプレート表面を経由せずに直接到達する
実体波は非常に弱く、散乱波の発生にもほとんど寄与しないと考えらる.
以上のことから,探触子からスリットへの直達波が計算上の入射場に
含まれないことは,次に示すRTMイメージング上で問題とならない.
\begin{figure}[htb]
\centering
	\includegraphics[clip,scale=0.4]{Figs/IncBtm.eps}
	\caption{計測波形から再構成した入射場(スリットがない場合)}
	\label{fig:inc_btm}
\end{figure}
%%%%%%%%%%%%%%%%%%%%%%%%%%%%%%%%%%%%%%%%%
\subsection{イメージング結果}
{\rm 図}-\ref{fig:inc_btm}の入射場と{\rm 図}-\ref{fig:snap_crack}の時間反転場の
FDTD解析結果から得られた$K_\rho(\fat{x}),K_{\lambda}(\fat{x})$および$K_{\mu}(\fat{x})
$を{図-}\ref{fig:imgs}に示す. この図では,3種類のRTMイメージをそれぞれの最大値で無次元化
して表示している.ただし,散乱源の存在は,画像化関数である$K_\rho,K_\lambda$および$K_\mu$
が正の値を取る箇所として示されるため,画像の表示範囲は0から1としている.
白の実線はFDTDモデル,すなわち,試験体の外形を示し,紫の実線は,次の{\rm 図}-\ref{fig:imgs_zoom}
で拡大して表示した範囲を表している.
なお,RTMイメージングは{\rm 図}-\ref{fig:fd_model})の解析領域全体で行った,
{図-}\ref{fig:imgs}に示す継手部周辺の外側では,画像化関数の値(画素値)は小さく,
有意な指示となって現れないことを確認している.
%
これら3種類のRTMイメージのうち$K_\lambda(\fat{x})$は,スリットと切り欠きの
周辺に比較的大きな画素値もつ部分が見られるものの,この結果からスリットの有無を
判断できるものではない.$K_{\lambda}(\fat{x})$はFWIの観点では,ラメ定数$\lambda$の
変動に対する目的関数の感度を意味する.$\lambda$は縦波速度に関係する弾性係数であるため,
縦波散乱波がスリットで強く励起されないことが$K_\lambda(\fat{x})$に指示が現れない原因
と考えられる.
%
一方,$K_\rho(\fat{x})$と$K_{\mu}(\fat{x})$では,切り欠き側面とスリット
開口部付近に大きな画素値をもつ箇所がある.{\rm 図}-\ref{fig:imgs_zoom}は
その状況をより詳しく観察するために,{\rm 図}-\ref{fig:imgs}において紫の実線で
囲った箇所を拡大したものである.これら拡大図ではイメージング結果とスリット
位置を対照するために,スリット位置を白の破線で描き加えてある.
{\rm 図}-\ref{fig:imgs_zoom}の(a)と(c)に示される通り,
$K_\rho$と$K_\mu$では大きな画素値をもつ位置が,実際の散乱源である
スリットと切り欠き側面に非常によく一致している.
一方で,スリットのおよそ下半分は画像に現れず,これはスリット先端側からの
散乱波が観測されていないことが原因と考えられる.
実際,{\rm 図}-\ref{fig:inc_btm}で入射場の挙動を見ると,プレート底面側から
くる横波はスリットの端部を逸れ,切り欠きのやや上側に向かうことが確認できる.
以上のことを考慮すれば,観測波形に含まれる微弱かつ複雑な経路をたどる横波散乱波が,
時間反転場の計算を通じて効果的に利用されていると言える.
ただし,相対画素値が0.2から0.4程度となる箇所はスリット近傍以外にも多数見られ
背景のノイズなっている.これは,継ぎ手内部の残響によるもので,完全に抑制
することは難しい.しかしながら,送受信位置と観測時間範囲の最適化や,
より受信感度の高いセンサーを計測に用いることで,信号−雑音比を向上
させることは十分に期待できる.
%
また,同様なイメージング結果を開口合成法のような波形の遅延−重ね合わせ
に基づく手法で得ることは以下の理由から困難で,このことからも時間反転法
の開発や超音波探傷への応用には意義あると言える.
第一に,開口合成法では散乱波の経路特定が必要とされるが、時間反転場
の計算を行うことなく,今回のような複数回の反射とモード変換を伴う
エコー経路を特定することは困難である.第二に,経路が正しく特定され場合
にも,対応するエコー波形をそのまま画像化領域に投影するイメージング方法
では、エコー継続時間がそのまま像の滲みとなって現れる。 
今回の計算で言えば,{\rm 図}-\ref{fig:snap_crack}(f)において赤色で
示された波群が占める程度の滲みは避けられず,RTMイメージのような
解像度を得ることは期待できない.さらに,画像解像度の向上には
デコンボリューションによるパルス圧縮で波形データを前処理する
ことが考えられる.しかしながら、デコンボリューションには,対象となる
波形と類似したスペクトル成分を持つ参照波形が必要とされる.
そのような波形を,今回のように波形が顕著に変化するエコーに対して
用意することには多くの困難が伴う.
%{図-}\ref{fig:imgs}に示したRTMイメージに改善の余地はあるものの,
以上の理由から,本研究の結果は,周辺境界からの反射,回折が無視できない
複雑形状部位の超音波イメージングにおいて,時間反転法が従来法に比べてより
有効なアプローチであることを示すと考えられる.
%同様な解像度や正確さをもつ画像を従来のSAFTや回折トモグラフィー的手法で実現することは難しく困難で,
\begin{figure}[htb]
\centering
	\includegraphics[clip,scale=0.6]{Figs/imgs.eps}
	\caption{計測波形を用いて行った3種類のRTMイメージングの結果}
	\label{fig:imgs}
\end{figure}
\begin{figure}
	%[htb]
\centering
	\includegraphics[clip,scale=0.6]{Figs/imgs_zoom.eps}
	\caption{計測波形を用いて行った3種類のRTMイメージングの結果(スリット周辺を拡大して表示)}
	\label{fig:imgs_zoom}
\end{figure}
